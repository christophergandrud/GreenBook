%%%%%%%%%%%%%%%%%%%%%%
% Does Presidential Partisanship Affect Fed Inflation Forecasts?
% Christopher Gandrud & Cassandra Grafström
% Updated 16 November 2012
%%%%%%%%%%%%%%%%%%%%%%

\documentclass{beamer}
\usetheme{MichiganYonsei}
\setbeamercovered{transparent}
\usepackage{color}
\usepackage{hyperref}
  \hypersetup{
		colorlinks=true
		linkcolor=black
		}
\usepackage{graphics}
\usepackage{booktabs}
\usepackage{url}
\newcommand{\comment}[1]{}


%%%%%%%%%%%%%%%%%%%%%%%%%%%%%%%% Title Slide %%%%%%%%%%%%%%%%%%%%%%%%%%
\title[Partisan Inflation Forecast Errors]{Does Presidential Partisanship Affect Fed Inflation Forecasts?}
\author[]{
    \href{mailto:gandrud@yonsei.ac.kr}{Christopher Gandrud} \and         \href{mailto:{cgrafstr@umich.edu}}{ Cassandra Grafstr\"{o}m}
}


\begin{document}

\frame{\titlepage}

\section[Outline]{}
\frame{\tableofcontents}

%%%%%%%%%%%%%%%%%%%%%%%%%%%%%%%%% Motivation %%%%%%%%%%%%%%%%%%%%%%%%%%%%%%%%%%%%%
\section{Motivation}
\frame{
  \frametitle{Working Paper}
  \begin{center}

    {\Large{The working paper is available on SSRN at: \\[0.5cm] \url{http://papers.ssrn.com/sol3/papers.cfm?abstract_id=2105301}.}}
  \end{center}
}

\frame{
  \frametitle{Presidential Partisan Inflation Forecast Bias}
  {\Large{Presidential Partisan Inflation Forecast Bias:}}\\[0.5cm] 
  When inflation forecasts are systematically different depending on the partisan identification of the United States president.
}

\frame{
    \frametitle{Motivation}
    {\LARGE{Why should we care about presidential partisan inflation forecast bias?}} \\[0.5cm]
        \begin{itemize}
            \item<1-> Clark \& Arel-Bundock (2011) find policymakers at the Federal Reserve are not politically indifferent.
            \item<2-> Could be that the information they receive is biased.
		\item<3-> Economists have not considered political preferences when evaluating Fed accuracy.
        \end{itemize}
}     

%%%%%%%%%%%%%%%%%%%%%%%%%%%%%%%%% Describing Forecast Errors %%%%%%%%%%%%%%%%%%%%%%%%%%%%%%%%%%%%%

\section{Describing Forecast Errors}
\frame{
    \frametitle{Forecast Accuracy}
        \begin{center}
            {\LARGE{How accurate are Fed inflation forecasts?}}
        \end{center}
}

\frame[plain]{
    \includegraphics[scale=0.65]{/git_repositories/GreenBook/Paper/figure/BaseInflation.pdf}
}

\frame{
  \frametitle{Forecast Errors}
  Our {\bf{dependent variable}}: \\[0.5cm]
\[
    E_{q} = \frac{F_{q} - I_{q}}{I_{q}}
\]
  \begin{itemize}
    \item<2-> $E_{q} = $ the standardized inflation forecast error for quarter $q$.
    \item<3-> $F_{q} = $ Green Book inflation forecast for quarter $q$. (We use forecasts made {\emph{two quarters}} prior).
    \item<4-> $I_{q} = $ actual inflation in quarter $q$.
  \end{itemize}
}

\frame{
  \frametitle{Forecast Errors}
  \begin{center}
    {\LARGE{Ideally, the mean forecast error is 0.}} \\[0.5cm]

	{\large{Consistent errors $\rightarrow$ ``wrong" policies.}}
  \end{center}
}


\frame{
  \frametitle{Traditional understanding of Fed forecasting}
\begin{itemize}
	\item<1-> Forecasts produced for every FOMC meeting.
	\item<2-> Product of both econometric models and expert judgments.
	\item<3-> Over long run no bias (e.g., Romer and Romer 2000).
	\item<4-> Periods of over- and under-estimations (Capistr\'{a}n 2008).
	\item<5-> No research on partisan influence of forecast errors.
\end{itemize}
}


\frame[plain]{
  \includegraphics[scale=0.65]{/git_repositories/GreenBook/Paper/figure/partisanError.pdf}
}

%%%%%%%%%%%%%%%%%%%%%%%%%% What might explain forecast errors? %%%%%%%%%%%%%%%%%%%%%%%%%%%%%%%

\section{What Might Explain Forecast Errors?}
\frame{
    \frametitle{Possible explanations}
        \begin{center}
            {\LARGE{What might explain forecast errors?}}
        \end{center}
}


\frame[plain]{
  \includegraphics[scale=0.4]{ExpectationsGraph.pdf}
}

%%%%%%%%%%%%%%%%%%%%%%%%%% Empirical Tests %%%%%%%%%%%%%%%%%%%%%%%%%%%%%%%


\section{Empirical Tests}
\frame{
  \frametitle{Empirical Analysis Overview}
    \begin{center}
      {\LARGE{Followed Ho et al. (2010) to isolate relationship between presidential partisanship/elections and the other controls.}} \\[0.5cm]
    \end{center}
    \begin{enumerate}
      \item<2-> Two data sets {\bf{matched}} on:
        \begin{itemize}
            \item<2-> {\emph{presidential party ID}}
            \item<2-> {\emph{election period}}
        \end{itemize}
      \item<3-> Used these in {\bf{parametric models}} with standardized inflation forecast errors as continuous dependent variable. 
    \end{enumerate}
}

\frame{
    \frametitle{}
        \begin{center}
            {\LARGE{Results?}}
        \end{center}
}

\frame[plain]{
  \begin{center}
    {\LARGE{Main Results}} (2 Quarter Old Forecasts)
  \includegraphics[scale=0.65]{/git_repositories/GreenBook/Paper/figure/CoefComparePlots.pdf}
  \end{center}
}

\frame[plain]{
  \begin{center}
    {\LARGE{Simulated Errors}} (All Forecasts)
    \includegraphics[scale=0.65]{/git_repositories/GreenBook/Paper/figure/ExpectValueParty.pdf}
  \end{center}
}

\frame[plain]{
  \begin{center}
    {\LARGE{Interactions}} (2 Quarter Old Forecasts)
  \end{center}
  \includegraphics[scale=0.65]{/git_repositories/GreenBook/Paper/figure/InterPlot.pdf}
}

\frame[plain]{
    \begin{center}
        {\LARGE{Diagnostic}} Orthogonal Dependent Variable \\[0.5cm]
        \includegraphics[scale=0.60]{/git_repositories/GreenBook/Paper/figure/GraphPartisanErrorUnemploy.pdf}
    \end{center}
}

%%%%%%%%%%%%%%%%%%%%%%%%%% Conclusions %%%%%%%%%%%%%%%%%%%%%%%%%%%%%%%

\section{Conclusions}
\frame{
    \frametitle{Conclusions}
    {\LARGE{Does presidential partisanship affect Fed staff inflation forecasts? \\[1cm]
        \begin{center}
            Probably.
        \end{center}
        }
    }
}

\frame{
  \frametitle{Conclusions}
  {\LARGE{How?}} \\[0.5cm]
  \begin{itemize}
    \item<1-> Fed staff {\bf{don't}} have an electoral bias. 
        \begin{itemize}
            \item<2-> Don't seem to try to influence election outcomes or compensate for FOMC political preferences.
        \end{itemize}
         \\[0.3cm]
    \item<3-> Fed staff {\bf{do}} use a {\bf{partisan heuristic}}. 
		\begin{itemize}
			\item Leads to {\bf{systematic bias}} in inflation forecasts across presidential terms.
		\end{itemize}
  \end{itemize}
}

\frame{
  \frametitle{Conclusions}
  {\LARGE{Possible political implications?}} \\[0.5cm]
  \begin{itemize}
    \item<1-> High inflation forecasts during {\bf{Democratic}} presidencies $\rightarrow$ interest rates {\bf{`too high'}}. 
	\begin{itemize}
		\item This could hurt Democrats' re-election chances.
	\end{itemize}
    \item<2-> Low forecasts during {\bf{Republican}} presidencies $\rightarrow$ interest rates {\bf{`too low'}}. 
	\begin{itemize}
		\item This could help Republicans' re-election chances.
	\end{itemize}
 \\[0.3cm]
	\item<3-> Does not explain Clark and Arel-Boondock's interest rate finding.
    \item<4-> Of course, {\bf{more research is needed}}.
  \end{itemize}
}


\frame{
	\frametitle{Backup Slides}
}


\frame{
	\frametitle{Propensity matching}
		\begin{center}
{\Large{Propensity Score Matching by Election Quarter}}
\includegraphics[scale=0.6]{/git_repositories/GreenBook/Paper/figure/ElectPropensity.pdf}
		\end{center}
}

\frame{
	\frametitle{Propensity matching}
		\begin{center}
{\Large{Propensity Score Matching by Presidential Party ID}}
\includegraphics[scale=0.6]{/git_repositories/GreenBook/Paper/figure/PresPropensity.pdf}
		\end{center}
}



\frame[plain]{
		{\Large{OLS Regressions with Non-Matched Data}}
\includegraphics[scale=0.45]{/git_repositories/GreenBook/Paper/figure/Table1.pdf}
}


\frame[plain]{
{\Large{OLS Regressions with Election Matched Data}}
\includegraphics[scale=0.45]{/git_repositories/GreenBook/Paper/figure/Table2.pdf}
}


\frame[plain]{
{\Large{OLS Regressions with Party Matched Data}}
\includegraphics[scale=0.45]{/git_repositories/GreenBook/Paper/figure/Table3.pdf}
}


\frame[plain]{
\begin{table}[ht]
\begin{center}
\caption{Bayesian Normal Linear Regression Estimation of Covariate Effects on 2 Qtr. Inflation Forecast Error (non-matched data set)}
\label{OutputNB}
{\small
\begin{tabular}{lccccc}
  \hline
Variables & Mean & SD & 2.5\% & 50\% & 97.5\% \\ 
  \hline
Intercept & 4.49 & 0.99 & 2.56 & 4.49 & 6.46 \\ 
  Pres. Party ID & 0.30 & 0.04 & 0.22 & 0.30 & 0.38 \\ 
  Recession & 0.07 & 0.05 & -0.04 & 0.07 & 0.17 \\ 
  Qtr. to Election & -0.00 & 0.00 & -0.01 & -0.00 & 0.00 \\ 
  Senate Dem/Rep & -0.26 & 0.15 & -0.56 & -0.26 & 0.05 \\ 
  House Dem/Rep & 0.16 & 0.13 & -0.09 & 0.16 & 0.41 \\ 
  Debt/GDP & 0.00 & 0.00 & -0.01 & 0.00 & 0.01 \\ 
  Expenditure/GDP & 0.12 & 0.04 & 0.05 & 0.12 & 0.19 \\ 
  Output Gap & -0.07 & 0.01 & -0.10 & -0.07 & -0.04 \\ 
  Discount Rate Change & -0.27 & 0.09 & -0.44 & -0.27 & -0.10 \\ 
  Global Model & -0.10 & 0.08 & -0.27 & -0.10 & 0.06 \\ 
  sigma2 & 0.04 & 0.00 & 0.03 & 0.03 & 0.04 \\ 
   \hline
\end{tabular}
}
\end{center}
\end{table}
}

\frame[plain]{
\begin{table}[ht]
\begin{center}
\caption{Bayesian Normal Linear Regression Estimation of Covariate Effects on 2 Qtr. Inflation Forecast Error (Matched by President's Party ID variable}
\label{OutputPB}
{\small
\begin{tabular}{lccccc}
  \hline
Variables & Mean & SD & 2.5\% & 50\% & 97.5\% \\ 
  \hline
Intercept & 4.60 & 3.74 & -2.70 & 4.59 & 11.90 \\ 
  Pres. Party ID & 0.34 & 0.08 & 0.19 & 0.34 & 0.49 \\ 
  Recession & 0.13 & 0.16 & -0.19 & 0.13 & 0.45 \\ 
  Qtr. to Election & 0.01 & 0.01 & -0.01 & 0.01 & 0.03 \\ 
  Senate Dem/Rep & -0.33 & 0.32 & -0.96 & -0.34 & 0.31 \\ 
  House Dem/Rep & 0.13 & 0.27 & -0.40 & 0.13 & 0.66 \\ 
  Debt/GDP & -0.00 & 0.01 & -0.02 & -0.00 & 0.01 \\ 
  Expenditure/GDP & 0.20 & 0.08 & 0.05 & 0.20 & 0.35 \\ 
  Output Gap & -0.08 & 0.05 & -0.18 & -0.08 & 0.01 \\ 
  Discount Rate Change & -0.46 & 0.34 & -1.12 & -0.46 & 0.20 \\ 
  Global Model & 0.02 & 0.15 & -0.27 & 0.02 & 0.31 \\ 
  sigma2 & 0.05 & 0.01 & 0.03 & 0.05 & 0.08 \\ 
   \hline
\end{tabular}
}
\end{center}
\end{table}
}
\end{document}