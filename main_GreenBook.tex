%%%%%%%%%%%%%%%%%%%%%%%%%%%
% Inflated Expectations
% Cassandra Grafström and Christopher Gandrud
% 17 August 2014
%%%%%%%%%%%%%%%%%%%%%%%%%%%%

% !Rnw weave = knitr

\documentclass[a4paper]{article}
\usepackage{fullpage}
\usepackage[authoryear]{natbib}
\usepackage{setspace}
    \doublespacing
\usepackage{hyperref}
\hypersetup{
    colorlinks,
    citecolor=black,
    filecolor=black,
    linkcolor=cyan,
    urlcolor=cyan
}
\usepackage{dcolumn}
\usepackage{booktabs}
\usepackage{url}
\usepackage{tikz}
\usepackage{todonotes}
\usepackage[utf8]{inputenc}
\usepackage{graphicx}

% Set knitr global options


%%%%%%% Title Page %%%%%%%%%%%%%%%%%%%%%%%%%%%%%%%%%%%%%%%%%%%%
\title{Inflated Expectations: How government partisanship shapes monetary policy bureaucrats' inflation forecasts}

\author{Christopher Gandrud \\
                {\emph{Hertie School of Governance}}\footnote{Post-doctoral Fellow. Friedrichstra{\ss}e 180. 10117 Berlin, Germany. Email: \href{mailto:gandrud@hertie-school.org}{gandrud@hertie-school.org}.} \\
                and \\
            Cassandra Grafstr\"{o}m \\
                {\emph{University of Michigan}}\footnote{Ph.D Candidate. 5700 Haven Hall, 505 S. State Street
                Ann Arbor, MI 48109-1045. USA. Email: \href{mailto:cgrafstr@umich.edu}{cgrafstr@umich.edu}.} \footnote{Thank you to the Mark Hallerberg and the Fiscal Governance Centre at the Hertie School of Governance for comments and support. Thank you also to Leonardo Baccini, Vincent Arel-Bundock, Mark Kayser, Cheryl Schonhardt-Bailey, Tom Stark, and seminar participants at the Hertie School of Governance, London School of Economics, and Yonsei University as well as two anonymous reviewers.}}

\begin{document}

\maketitle

%%%%%%% Abstract %%%%%%%%%%%%%%%%%%%%%%%%%%%%%%%%%%%%%%%%%%%%
\begin{abstract}

\emph{Forthcoming in Political Science Research and Methods}

Governments' party identifications can indicate the types of economic policies they are likely to pursue. A common rule of thumb is that left-party governments are expected to pursue policies for lower unemployment, but which may cause inflation. Right-party governments are expected to pursue lower inflation policies. How do these expectations shape the inflation forecasts of monetary policy bureaucrats? If there is a mismatch between the policies bureaucrats \emph{expect} governments to implement and those that they \emph{actually} do, forecasts will be systematically biased. Using US Federal Reserve Staff’s forecasts we test for executive partisan biases. We find that irrespective of actual policy and economic conditions forecasters systematically overestimate future inflation during left-party presidencies and underestimate future inflation during right-party ones. Our findings suggest that partisan heuristics play an important part in monetary policy bureaucrats' inflation expectations.

\end{abstract}

\begin{description}
  \item [{\textbf{Keywords:}}] forecast bias, Federal Reserve, bureaucrats, rational partisan cycle, heuristics, inflation, monetary policy
\end{description}

\vspace{0.3cm}


Monetary policy is an inherently forward looking enterprise. Beliefs about the economy's future course significantly guide the setting of interest rates \citep[59]{Goodhart2001}. Government policies as diverse as tax, spending, and regulatory policies have important effects on changes in growth, inflation, and unemployment. Further, a government's party identification can serve as a cue for the types of economic policies that it is likely to pursue during its tenure. A common expectation for the United States is that right-leaning Republicans will pursue policies associated with lower inflation and left-leaning Democrats will pursue policies associated with lower unemployment, but also more inflation \cite[see][]{Samuelson1977,HibbsJr1977}. Recent evidence suggests, however, that the real differences in economic policies implemented by the two parties are quite minimal \citep{Bartels2008}. Pursuing monetary policy based on expectations of differences in partisan behavior, as opposed to the reality of their general similarities, could lead to systematic mistakes in the setting of monetary policy. These mistakes could exacerbate fluctuations in the business cycle by allowing output and unemployment to remain too low when inflation expectations are higher than inflation turns out to be and allowing bubbles to form when inflation expectations are too low. It is therefore important to ask how US Federal Reserve Staff incorporate the government's partisan composition when forming expectations of future inflation.

We provide strong evidence that Fed internal inflation forecasts--the forecasts on which monetary policy decisions are based \citep[130]{Adolph2013}--are heavily influenced by an inaccurate presidential partisan rule of thumb or `heuristic'. These forecasts consistently predict that inflation will be lower than it turns out to be under Republican presidencies by about 11 percent on average and that inflation will be higher than it turns out to be under Democratic administrations by about 13 percent. Even accounting for changes in monetary policy and a variety of other economic and political factors, Federal Reserve economists over-shoot inflation forecasts for Democrats and under-shoot for Republicans. We find that previous literature on how partisanship may affect monetary policy outcomes--primarily work done on partisan preferences \citep{Clark2012,Hakes1988,Sieg1997,Tootell1996} and rational monetary policy expectations \citep{Alesina1987,Alesina1991,Hibbs1994}--is not as useful as a partisan heuristic approach for explaining these predictive failures by Fed Staffers.

In this paper we first briefly describe bureaucratic inflation forecasting at the US Federal Reserve, why it is important for monetary policy-making, and previous research on sources of bias in it. Since academic scholarship on partisanship and Fed forecasting has largely been non-existent we introduce the presidential partisan heuristic approach to explain prediction errors. We also derive major alternative hypotheses about how partisan control of the presidency might shape inflation forecasts by Federal Reserve Staff from the literature on partisanship and monetary policy-making. We then discuss how to measure inflation forecast errors. Using Fed Staff's ``Greenbook'' forecasts we demonstrate that there does appear to be a presidential partisan bias. To understand why these errors exist we test the theories of partisan bias with a series of regression models with data on inflation forecast errors from the 1970s through 2007.\footnote{This is the most complete data set currently available to the public.} Our findings suggest that even when controlling for a number of important economic and political factors, Greenbook forecasts show a distinct presidential partisan bias across presidential terms, not just in the run up to elections, as competing theories would suggest. Rather than being caused by Staff and/or monetary policy-makers' electoral preferences or partisan monetary policy expectations--we find strong evidence supporting our hypothesis that the bias is caused by an incorrect partisan heuristic Fed Staff hold about presidential administrations' likely effects on inflation. Our finding highlights the interaction between political and psychological aspects of how bureaucrats deal with uncertainty that have previously been ignored by researchers examining monetary policy bureaucracies. In the conclusion we discuss the implications of these findings for monetary policy, election outcomes, and future research directions.

%%%%%%%%%%%%%% Section 1: Forecasting Inflation at the Fed %%%%%%%%%%%%%%%%%%%%%

\section{Bureaucratic Inflation Forecasting in the United States}

Inflation forecasting is crucial for enabling monetary policy-makers to maintain price stability. The primary instrument of monetary policy--interest rates--``only exert their full effect on \ldots inflation with some considerable delay'' \citep[59]{Goodhart2001}. The US Federal Reserve's Greenbook forecasts are an important component of how monetary policy-makers' predict future inflation in the United States. Prior to every Federal Open Markets Committee (FOMC) meeting Federal Reserve Staff\footnote{Specifically, economists at the Washington, D.C. Federal Reserve Board generate the estimates contained in the Greenbook. Regional banks also create their own sets of estimates for both their region and the country as a whole. Unfortunately, these estimates are not universally available from the regions.} create a document called the ``Current Economic and Financial Conditions''--affectionately called the Greenbook due to the color of its cover--that contains information on recent behavior and forecasts of various macroeconomic aggregates assuming no monetary policy change.\footnote{Greenbook data can be found at {\url{http://www.phil.frb.org/research-and-data/real-time-center/greenbook-data/philadelphia-data-set.cfm}} (accessed March 2013). Greenbook forecasts are currently available to the public for each quarter from the fourth quarter of 1965 through the end of 2007. There is a five year lagged release schedule. Also, some forecasts are not available for the entire period.} Federal Reserve Staff make forecasts of various elements of the US and global economies so that the FOMC can make policies appropriate to fulfill the Fed's dual mandate of maintaining maximum employment and price stability.

As \cite{Svensson2005} notes, the accuracy of forecasts is essential to the effectiveness of monetary policy. The FOMC directly uses these forecasts to determine the appropriate monetary policy to pursue and they, ``have a large effect on the interest rate chosen by the Fed'' \citep[130]{Adolph2013}. Greenbook forecasts are given to FOMC members one week before each meeting. Staff also present the Greenbook forecasts during FOMC meetings. Expectations, directly influenced by Greenbook forecasts, play a very important part in FOMC decision-making. From FOMC minutes we know that members spend a considerable amount of time discussing prospective economic conditions. In fact, much of the FOMC meeting time is used to discuss what economic conditions are likely to be rather than the relative desirability of various outcomes \citep[230]{RomerRomer2008}. Greenbook forecasts directly frame these discussions. What members believe will happen in the future directly influences their policy choices. Higher inflationary expectations increase the likelihood of a member supporting raising interest rates in order to slow inflation and an overheating economy; low inflationary expectations increase the likelihood of a preference for lowering interest rates to bolster growth and employment, all else equal. Therefore, Greenbook inflation forecast accuracy is essential for FOMC members choosing optimal monetary policies.

The study of Fed inflation forecasts and their accuracy has been almost exclusively contained within economics, with the main concern being the performance of the Fed's forecasts relative to market forecasts \cite[e.g.][]{Romer2000,Faust2007, Gamber2009}. While some studies in the economics literature have examined the biases of particular time periods \cite[e.g.][]{Capistran2006} or Fed presidents \cite[e.g.][]{Havrilesky1995}, in our search none considered how government partisanship affects expectations about future inflation at the Federal Reserve.

It is important to note that finalized Greenbook forecasts are ``consensus'' forecasts combining both econometric models {\emph{and}} the professional opinions of forecasters about likely changes in the economy's trajectory missed in these models \citep{Karamouzis1989,Reifschneider1997}. Preferences and/or beliefs about government partisanship, rather than just econometric models based on explicit assumptions, could therefore directly shape Greenbook forecasts.\footnote{Unfortunately, only consensus forecasts are released publicly. It is unfortunately impossible to separately observe the model and judgmental components of the consensus forecasts.}

The idea that politicians of different partisan stripes might behave differently in office and that their behavior might have different effects on future economic output and inflation is largely uncontroversial for political scientists.\footnote{See \cite{Franzese2002,Bartels2008} for evidence on the similarities and differences of Democrats and Republicans in office.} Scholars have long argued that left- and right-wing politicians pursue policies that have very different effects on inflation and unemployment. The working-class electoral base of left-wing parties leads them to pursue policies that increase employment, while right-wing parties' dependence on business-owners for political support leads them to pursue policies that reduce inflation \citep[e.g.][]{MacRae1977,Nordhaus1975,Schultz1995,Tufte1980}. However, political scientists have largely not explored whether or how these differences affect Federal Reserve inflation forecasts. The closest attempt made to explore partisan biases in Fed forecasts that we know of was done by \cite{Frendreis2000}. Using simple frequency tables and yearly data, they examined the accuracy of forecasts by the Congressional Budget Office (CBO), presidential administrations', and Federal Reserve Staff. Though they listed the accuracy of Fed inflation forecasts as measured by absolute mean error for the whole period 1979-1997,\footnote{They found that absolute mean errors were similar to the CBO's and less than administrations'.} they did not examine partisan biases or any other cause of inaccuracies in Fed forecasts. Their study of partisan biases was entirely confined to a comparison of CBO and administrations' forecasts.

%%%%%%%%%%%%%%%%%%%%%%%%%%%%% Section 2: Possible Explanations of Fed Inflation Forecast (In)accuracy %%%%%%%%%
\section{Possible Explanations of Fed Inflation Forecast Inaccuracy}

We begin the first significant investigation of possible partisan biases in Fed Staff's inflation forecasting errors in this section by introducing our presidential partisan heuristic approach. A theory is useful only to the extent that it explains a phenomenon better than the most plausible alternatives. So, after introducing partisan heuristics, we posit alternative explanations for presidential partisan based errors in inflation forecasts that build on previous approaches to understanding the relationship between partisanship and monetary policy decisions. We reformulate these for the issue of bureaucratic inflation forecasting. Key difference between each of these theories are assumptions about how accurately forecasters' can predict inflation given available information and what they assume motivates forecasters: forecast accuracy, partisan electoral success, and/or maintaining stable prices.

\subsection{Presidential Partisan Heuristics}

Previous research has shown that (a) central bankers use heuristics and similar judgmental `rules of thumb'--such as Okun's law and the Taylor Rule--to help them understand complex and uncertain economic conditions and (b) that people's economic expectations are shaped by the government's partisan orientation. To understand how presidential partisanship may impact inflation forecasting errors we propose combining these two insights into one coherent approach that we call {\bf{presidential partisan heuristics}}. Let's look at each part of this theory in turn.

\paragraph{Heuristics \& inflation expectations}

We begin with the assumption that Fed Staff have an interest and/or incentive (e.g. because of professional socialization, performance reviews, and regular academic analyses that possibly could provoke FOMC or even Congressional scrutiny) to make the most accurate forecasts possible. This is a reasonable assumption given the numerous studies that have found Fed Staff forecasts to be more accurate than private sector and other government agency forecasts \cite[see][]{Frendreis2000,Romer2000,Gamber2009}.

However, even if we assume that Fed Staff want to make very accurate predictions there is still considerable uncertainty about future inflation, the effects of monetary policy on inflation due to the complexity of economic relationships that produce inflation, and the difficulty of adequately observing these relationships \cite[see][22-24]{Schonhardt2013}. In fact, \cite{Gamber2009} find that though the overall level of inflation in the United States has decreased since the early 1980s, inflation has actually become more difficult to forecast. They argue that this is because the processes underlying inflation became more unpredictable during the period known as the `Great Moderation' from about the mid-1980s until 2008. Thus, even if Fed Staff want to accurately predict inflation, it is difficult for them to do so. This is precisely the type of situation where we would expect actors to supplement their forecasting models with heuristic judgments in an attempt to create better expectations.

Heuristics are intuitions that reduce the uncertainty associated with making predictions. Considerable psychological and economic research has shown across a wide range of settings, including among expert forecasters, that people reduce the complexity of predicting uncertain values by using simple heuristics \citep[see][]{kahneman1973,tverskykahneman1974,Tversky1983,Kahneman2002,kahneman2003}. For example, \cite{Gray2013} notes that there is uncertainty about whether or not an emerging market sovereign will service their debt. To overcome this uncertainty, she finds that foreign investors use international organization membership as a heuristic for whether or not a country is likely to pay them back. Countries that join organizations with reputable members are viewed as less risky and vice versa.

The role of judgment and heuristics to help forecast inflation in light of its complexity and uncertainty has been widely discussed and researched among Federal Reserve Staff members themselves. For example, \cite{McNees1990} used individual forecasts to examine situations under which forecasters' judgmental adjustments improved (or didn't) economic forecasts. He found that though judgment can add important information to forecasts, forecasters have a tendency to over adjust their models based on their own judgments. \cite{Orphanides2008} found that FOMC members used a Taylor ``rule of thumb'' with expected economic conditions when making interest rate decisions. Work by \cite{Tillmann2010Philips} and \cite{KnotekII2007} has found that FOMC members use Phillip's Curve and Okun's Law rules of thumb to predict the relationship between unemployment, inflation, and growth. Overall, the focus in this work has been on rules of thumb based on economic, rather than political factors.

\paragraph{Partisan inflation expectations}

However, inflation is not a purely economic phenomenon, in the sense that government policies beyond monetary policy--narrowly defined as setting interest rates--do affect future inflation. Partisan theories predict that these policies will vary systematically with government partisanship \citep{Samuelson1977,HibbsJr1977}. One of the main assertions of these theories is that left-leaning Democratic politicians will pursue policies that reduce unemployment and increase inflation, while right-leaning politicians will pursue policies that reduce inflation even at the cost of higher unemployment \citep[e.g.,][]{MacRae1977,Nordhaus1975,Schultz1995,Tufte1980}. The electoral base of the Democratic Party has historically been working class voters who are most negatively affected by unemployment. They also own fewer assets that depreciate with inflation. The financial interests of the Democratic base produces incentives for Democratic politicians to pursue pro-employment policies, even at the cost of higher inflation. The Republican base historically consists of wealthier individuals and business owners, for whom increased inflation has a significant negative impact on their wealth and production costs; but who are less likely to be directly affected by unemployment \citep{Hibbs1987}. These voter preferences produce the expectation that right-wing politicians will enact policies aimed at containing inflation \citep[e.g.,][]{Chappell1986}. Furthermore, these issues are effectively ``owned" by the parties, with Democrats considered the caretakers of employment and Republicans the spending and inflation watchdogs, even by those who do not identify as their supporters \citep[c.f.][]{Petrocik2003}. Given these common beliefs, it would be reasonable to include information about what policies a government is expected to pursue when forecasting inflation.

To our knowledge the incorporation of information on government partisanship into monetary bureaucrats' inflation expectations has not been examined or explicitly discussed by Fed forecasters. However, previous research has found that members of the public do use partisan heuristics to predict future economic conditions, including inflation. Examining survey data of general assessments of economic health, \cite{Duch2000} found that people perceived both current and future economic conditions differently based on whether or not they shared a partisan affinity with the government. More pertinent to our discussion here, \cite{Snowberg2007} used data from presidential election prediction markets before the 2004 election to examine how partisan expectations move economic indicators. They found that a number of indicators, such as bond yields and exchange rates, moved in directions consistent with standard partisan expectations of policy changes under Democratic and Republican presidents in situations when it was believed that candidates from these parties would win.\footnote{Unfortunately, they did not have an economic indicator that could capture changes in inflation expectations.} \cite{Fowler2006} looked specifically at how prediction markets affected futures markets for nominal interest rates. When the probability of a Democrat winning the US Presidency (or Congress) increased, so did interest rate futures. He interprets this finding to suggest that actors expected inflation to increase under a Democratic government. This confirmed a finding using similar methods by \cite{Alesina1997}. \cite{Berlemann2006} extend evidence for a partisan inflation expectations effect to five other developed countries in addition to the United States.

It is important to  note that these pieces of research have found evidence that people {\emph{expect}} economic conditions to be different under different presidents, not that it actually is.\footnote{This line of reasoning might lead one to expect that the Fed may be responding to the public's beliefs about the inflationary impact of presidential partisanship. To the extent that the public believes a Democratic president will produce higher inflation than a Republican president, economic theory predicts this will produce higher inflation under Democratic presidencies. If the Fed were rationally accounting for the publics beliefs and subsequent behavior, they would adjust their inflation predictions accordingly. However, even if they do take these expectations into account, any errors in inflation forecasts  indicate that Staffers are over- or under-estimating the public's partisan inflationary expectations in a non-rational manner. We thank an anonymous reviewer for bringing up this point.}

\paragraph{Presidential partisan heuristics \& inflation forecasting errors}

Drawing on the findings that monetary policy-makers use rules of thumb and that presidents from different parties are widely believed to have different impacts on inflation we argue that Fed Staff forecasters are likely to incorporate heuristics about the expected effect of public policies on inflation based on the president's party identification. If these expectations do not accurately correspond to actual differences, then we will observe systematic forecasting errors across presidential terms.

Why focus on \emph{presidents'} party identifications? Incorporating information about expected non-monetary policy actions is particularly difficult in the American context. There are many layers of government--national, state, city--making policies that can impact inflation. Even within the national government there are multiple branches often controlled by different parties that can influence policies. Given this complexity forecasters may intuitively use a heuristic based on the partisan identification of, arguably, the single most influential actor in this system--the president--as a way to simplify the complex relationships between politics and inflation. The research discussed in the previous section substantiates this focus by finding that members of the public's expectations are affected by the president's party identification.

Presidential partisanship could be used as a ``prototype heuristic'' to help forecasters predict inflation in light of uncertainty. Prototype heuristics are a general class of heuristic where people substitute the mean or exemplar attribute--prototype--of a category for what they are trying to determine. Prototypes have been found to impact judgments on problems as diverse as the pricing of goods, the effectiveness of painful medical procedures, and the prediction of floods by professional forecasters \cite[]{kahneman2003}. Presidents' partisan identifications are easily observable and available on an intuitive level--i.e. they don't require conscious thought to remember. The probable impact of expected public policies on inflation is conversely uncertain. Fed Staff could use prototypical information about presidential party ID, which is intuitive and readily available, as a substitute for more complete, but more difficult to obtain information about what public policy is likely to be and how it will impact inflation.

As discussed above, the prototypical Democratic president is believed to be less concerned with price stability than the prototypical Republican president. The prototypical Republican president therefore pursues policies that dampen inflation and vice versa for the prototypical Democrat. Forecasters using a presidential partisan prototype heuristic would predict inflation to be higher under a Democratic president and lower under a Republican, all else equal.

Though heuristics can be useful, ``sometimes they lead to severe and systematic errors'' \citep[][1124]{tverskykahneman1974}. In our theory, economists at the Fed have an intuition that Democrats and Republicans behave differently in government and so formulate inflation expectations with this in mind. If this intuition does not accurately correspond to how presidents act, or how their policies impact inflation, forecasts will be systematically biased: overestimates for Democratic presidents, underestimates for Republicans. \cite{Bartels2008} finds evidence that Democratic and Republican presidents do not, in fact, have significantly different spending policies, so any expectation that they would pursue policies that would differentially affect inflation in the medium-run would likely be inaccurate.\footnote{\cite{Bartels2008} does not discuss other policies that could affect inflation in the long run, such as changes to labor and financial market regulation. These policies too would be expected to differ by party, however their lags are likely to be quite long and out of the forecasted time frames used in this paper. More specifically, \cite{Franzese2002} finds only moderate evidence for partisan monetary policy differences confined primarily to the period 1973-1982.} Biases about partisan effects on inflation should therefore be constant {\emph{throughout a president's term}}. As we will see in the next section, this prediction contrasts with the alternative arguments--partisan preference and monetary expectation theories. Both predict an intensification of biased forecasts as elections approach. Figure \ref{ExpectGraphs} shows the anticipated empirical patterns of inflation forecast errors over presidential terms by partisanship in the three theories we set out.

It is important to note that in contrast to typical rational partisan expectations approaches, our model does not require that forecasters be conscious of the heuristic they're using. It can simply work its way subtly into forecasts, particularly in the subjective component of the Greenbook's ``consensus forecast". If the models do not conform with other expectations about the economy's current course, based in part on these subtle partisan heuristics, the consensus forecast will be modified accordingly. Further, because this bias would not need to be conscious, the systematic error could easily go unnoticed (as mistakes could occur for any number of idiosyncratic or economic reasons). If the bias goes unnoticed, then it will not be corrected in future inflation predictions.\footnote{This assumption is in contrast to \cite{Grauwe2011}, who assumes that actors actively observe their heuristics and adapt them through trial and error. He does not provide empirical evidence supporting this assumption, however.} This differs from the rational partisan expectations theory \cite[e.g.][]{Alesina1987,Alesina1991,Alesina1997,Hibbs1994}. First, because these beliefs are not updated to account for a lack of partisan inflation differences they are not ``rational". Second, and relatedly, this theory is based on psychological instead of game theoretic reasoning, which allows for the persistence of sub-optimal strategies in a way that would be less likely in a rational choice model of this same process given the assumption that the goals of the actors are the same in the two models. Because errors arise due to stochastic processes in addition to the systematic heuristic biases described here, the failure to perfectly predict inflation would not necessarily lead Fed Staff to adjust their expectations or recognize their biases.




% Define colors for figure
%% See: http://colorbrewer2.org/
\definecolor{DEM}{HTML}{2259B3}
\definecolor{REP}{HTML}{C42B00}

\begin{figure}
  \caption{Stylized Partisan Inflation Forecast Error Predictions}
  \label{ExpectGraphs}
  \begin{center}

    \vspace{0.25cm}

    \tikzstyle{bagMain} = [text width = 5cm]
    \tikzstyle{bagDem} = [text = DEM]
    \tikzstyle{bagRep} = [text = REP]
    
    \tikzstyle{DemLine} = [draw, 
                          color=DEM,
                          opacity=0.9,
                          line width=1.5mm]

    \tikzstyle{RepLine} = [draw, 
                          color=REP,
                          opacity=0.9,
                          line width=1.5mm]

\begin{tikzpicture}

  %%%% Partisan Heuristics
  \node (PP) at (-7, 5) [bagMain]{{\bf{Partisan Heuristics}}};
  \node (E) at (-10.5, 3.25) [bagMain, rotate=90]{{\emph{Forecast Error}}};
  \node (T) at (-7.3, -0.5) [bagMain]{{\emph{Duration of Pres. Term}}};
  
  \draw (-10, 0) -- (-6, 0);
  \draw (-10, 0) -- (-10, 4);


  \draw[DemLine] (-9.5, 3) -- (-6.5, 3); 
  \draw[RepLine] (-9.5, 1) -- (-6.5, 1); 

  \node (R1) at (-6.5, 0.5) [bagRep]{Rep.};
  \node (D1) at (-6.5, 3.5) [bagDem]{Dem.};
  
  %%%% Partisan Preferences
  \node (PP) at (-2, 5) [bagMain]{{\bf{Partisan Preferences}}};
  
  \draw (-5, 0) -- (-1, 0);
  \draw (-5, 0) -- (-5, 4);
  
  \draw[RepLine] (-4.5, 1.9) -- (-2.5, 1.9); 
  \draw[DemLine] (-4.5, 2.1) -- (-2.5, 2.1); 

  \draw[RepLine] (-2.52, 1.9) -- (-1.5, 1); 
  \draw[DemLine] (-2.52, 2.1) -- (-1.5, 3); 


  \node (D2) at (-1.5, 0.5) [bagRep]{Rep.};
  \node (R2) at (-1.5, 3.5) [bagDem]{Dem.};
  
  
  %%%% Monetary Expectations
  
  \node (PP) at (3, 5) [bagMain]{{\bf{Monetary Expectations}}};
  
  \draw (0, 0) -- (4, 0);
  \draw (0, 0) -- (0, 4);

  \draw[RepLine] (0.5, 2.1) -- (2.53, 2.1); 
  \draw[DemLine] (0.5, 1.9) -- (2.53, 1.9); 

  \draw[DemLine] (2.53, 1.9) -- (3.23, 1); 
  \draw[RepLine] (2.53, 2.1) -- (3.23, 3); 

  \node (D3) at (3.5, 3.5) [bagRep]{Rep.};
  \node (R3) at (3.5, 0.5) [bagDem]{Dem.};
  


  \end{tikzpicture}
  \end{center}
\end{figure}



%%%%%%%%%%%%%%%

\subsection{Alternative partisan explanations}

The usefulness of a theory is partially demonstrated by how well it explains outcomes relative to its major competitors. Though no previous studies have examined partisan biases in inflation forecasts, competing theories can be derived from studies that have looked for evidence of partisan preferences manifesting themselves in the FOMC's monetary policy outcomes. Two key strains in this literature consider partisan effects as either resulting from a preference for one party over another by members of the FOMC (not from inaccurate underlying assumptions about the behavior of partisans) or an expectation that, once in office, the parties will engage in systematically different policies that will influence inflation, leading the FOMC to support more preferred policies and attempt to inhibit less preferred ones.

The preference arguments about monetary policy-making assume that a conservative central banker\footnote{That central bank economists would be more conservative than the average citizen is both a fundamental assumption of the central bank independence literature \citep[e.g.][]{Goodman1991} and backed by empirical evidence \citep[e.g.,][]{Scott1975,Stigler1959}.} will prefer the election of politicians who hold more similar inflationary preferences (i.e., those with a stronger preference for low inflation) and enact policies to bolster their preferred candidate's prospects of being elected. In the US, this would mean that the FOMC would implement policies that supported the electoral prospects of Republican incumbents and harm the electoral prospects of Democratic incumbents \citep{Clark2012,Hakes1988,Sieg1997,Tootell1996}.

Building on this approach, a {\bf{partisan preference theory}} of inflation forecast errors assumes that Fed Staff have a preference that more inflation averse politicians to control the executive and so produce inflation forecasts that would justify the implementation of easy monetary policy under Republican administrations and tight money under Democratic administrations, particularly as presidential elections approach. The FOMC, choosing policy based on these forecasts, would then implement monetary policies to optimize its utility function over low inflation and high employment, which would not need to depend upon presidential partisanship at the level of the FOMC.\footnote{It could be that the FOMC either is actually influencing Greenbook forecasts to justify interest rate changes that aid their preferred candidate before elections or that the FOMC and Fed Staff have identical partisan preferences. These would all be observationally equivalent in the absence of detailed case study work. However, as we discuss below, we find no evidence that partisan errors actually change in the run up to elections. As such we do not feel that further case study work on this issue would be useful at this point.} However, because Fed Staffers prefer low inflation to high, they would not necessarily want to produce too loose/tight monetary policy over an entire four year term. Instead, they would want to encourage an economic boost (contraction) near the end of a Republican (Democratic) presidency. This implies that realized average inflation would be higher than forecasted during Republican presidencies and lower than forecasted for Democratic presidencies. However, this relationship would be particularly pronounced in the {\emph{quarters running-up to elections}} as Fed Staff attempt to help their favored political party \citep{Beck1987,Grier1987}. Further, accounting for actual changes in monetary policy ought to increase the magnitude of partisan effects. This is because predictions of inflation during Republican presidencies, for example, will be lower than what the Staff actually expects. If looser monetary policy is implemented during electoral periods in response to these low inflation forecasts than would have been chosen under the Staff's true inflationary expectations, inflation will actually be higher than the Staff's true beliefs about inflation under no change in monetary policy.

An alternative set of existing theories--the rational partisan expectations literature on monetary policy-making--assumes that central bankers do not have an innate preference for one party over another, but instead accurately expect Democrats and Republicans to behave differently in office \citep{Alesina1991,Hibbs1994}. It is these behavioral expectations that would lead to different monetary policies under Democratic and Republican presidencies, with the former expected to engage in more expansionary and inflationary policies than the latter. In order to stave off higher inflation under a Democrat, the Fed would tighten monetary policy; because Republicans are expected to prefer lower inflation, they will pursue policies in support of that goal and so the FOMC can accommodate Republican presidents' policies without fear of stoking inflation. This argument is again based on the assumed preferences of partisans, but does not require the FOMC to be politically biased as the former does.

Building on the rational partisan expectations literature, what we call the {\bf{monetary expectations theory}} is in some ways the reverse of partisan preferences theory described above: it is based on an assumption of partisan bias in the FOMC, but {\emph{not}} the Staff. The Staff are assumed to have only a price stability preference. Federal Reserve economists believe members of the FOMC will engage in partisan monetary policy by lowering interest rates under right-leaning administrations in the run up to elections, and increasing them under left-leaning presidents, as \cite{Clark2012} found. The monetary expectations theory assumes that the FOMC is doing this to manipulate election outcomes. In this formulation, the Fed Staff has no preference for one party over another, but knows that the FOMC does, and so formulates estimates aimed at countering the FOMC's policies. If Fed economists believe that the FOMC will choose systematically higher-than-called-for interest rates during Democratic presidencies and vice versa for Republicans, then--assuming they are interested in the implementation of optimal monetary policies--they would produce forecasts that are higher than expected during Republican administrations and lower for Democrats in order to produce more optimal policies; the {\emph{opposite of what is expected in the partisan preference theory}}. If the FOMC fails to note the compensation made by the Fed Staff, then we would expect that after accounting for implemented policies inflation forecasts would be higher than or equal to realized inflation during Republican terms and lower than or equal to forecasts under Democratic administrations.\footnote{This is illustrated in the center panel of Figure \ref{ExpectGraphs}.} If, however, the FOMC anticipated these compensatory biases in Staff forecasts, then the FOMC would discount the Greenbook estimates and continue to implement inflationary policies during Republican administrations and contractionary policies during Democratic ones. If the Staff likewise know that they are not being listened to they may randomize their errors, producing an uninformative signal \citep{Crawford1982}. This would result in approximately similar inflation forecast errors for both Republicans and Democrats. However, we largely did not observe this (see below). If the Fed Staff believes that the FOMC will engage in partisan pumping only when presidential elections are approaching, then we would expect no partisan differences in forecasts at the beginning of a presidency but increasing divergence as the term wanes.

The monetary expectations theory models Fed Staff forecasts as partially a function of Staff interactions with the FOMC. What if Greenbook forecasts are even more directly influenced by FOMC members in that members' judgments are directly incorporated into the forecasts? This would have an important substantive implication. All of the partisan theories we have discussed have important policy implications to the extent that they effect FOMC inflation expectations and therefore their interest rate choices. What if this is backward? Why couldn't we assume that it is the FOMC members' partisan biases or preferences that are influencing staff forecasts?

It is very difficult to empirically determine if and to what extent informal discussions between Fed Staff and FOMC members influence Greenbook forecasts, because these discussions are not observable. However, the formal forecasting process as well as comparisons between Staff and FOMC members' forecasts suggests that the bulk of the influence runs from the Staff to the members, rather than the opposite direction.

Greenbook forecasts are presented to FOMC members before FOMC meetings, where expectations are debated at length, and before members make there own biannual formal forecasts.\footnote{Members are required by the 1978 Humphrey-Hawkins Act to make formal forecasts.} If Greenbook forecasts were simply parroting FOMC members' prior expectations then we would expect the two sets of forecasts to be very similar. This has not been the case. \cite{RomerRomer2008} found that FOMC members' forecasts are different from Greenbook forecasts and may in fact be less accurate predictions of inflation. Given that Greenbook forecasts are presented to members directly before FOMC meetings, forecasts act as a reference point from which FOMC members build their own expectations. If attempts by FOMC members to change Greenbook forecasts tend to result in less accurate forecasts then it is especially important that the reference point be as accurate as possible.

\subsection{Forecasting model accuracy}

Finally, before empirically digging into partisan explanations of forecast errors, which would largely be the result of Federal Reserve Staff judgment, it is worth examining the possibility that forecast inaccuracy is the result of \textbf{systematic errors in the Staff's predictive econometric models}. Federal Reserve Staff have primarily used two sets of econometric models during the period for which Greenbook data is available.\footnote{This discussion draws heavily on Brayton et al.'s \citeyear{Brayton1997} detailed description of the changes to Federal Reserve forecasting models that took place in 1996.}

The first simultaneous equation models of the US and world economies were developed and adopted by the Federal Reserve between 1966 and 1975. These models were based on adaptive expectations and largely extrapolated future behavior of the economy from its recent past behavior. New models of the American and world economies' near-term trajectories were introduced in the 1990s, fully replacing the older models in 1996. The Federal Reserve Board US model (FRB/US) and its counterpart for the global economy (FRB/Global) explicitly consider the role of economic expectations in economic behavior. The foundational assumption of adaptive expectations in the old models was replaced with rational or model-consistent expectations. In these models prices are sticky and aggregate demand determines short-run output. Furthermore, monetary policy's effects on the economy are extensively modeled. For a more detailed history, please see the discussion in the Supplementary Materials.

Presumably, the move to rational expectations would improve forecast accuracy relative to the earlier period. The goal of incorporating forward looking actors into the models was to account for an important source of endogeneity in earlier models that could lead to overestimates of important economic indicators under some circumstances and underestimates of those same indicators under others. None of these over or underestimates, however, ought to have been linked to the party of the president. We would, however, expect that the \emph{magnitude} of forecast errors shrank after 1996.\footnote{Unfortunately, we cannot more directly observe model errors. Only the consensus forecasts, i.e. model forecasts combined with judgmental adjustments, are made publicly available.}



%%%%%%%%%%%%%%%%%% Section 3: Forecast Accuracy %%%%%%%%%%%%%%%%%%
\section{Federal Reserve Staff's Forecast Accuracy}\label{ForecastAcc}

How accurate are Fed Staff forecasts? We focus on Greenbook forecasts of the GNP/GDP price index. We choose this indicator of Federal Reserve forecast accuracy because central bankers are believed to be primarily concerned with inflation \citep[e.g.][]{Cukierman1992,Mukherjee2008,Tillmann2008}. It is also the dominant measure of forecast errors used in the economics literature \citep[c.f.][]{Romer2000}.

We measure accuracy by calculating {\bf{forecast error}} $E$ as the difference between the Greenbook inflation forecast $F$ for a given quarter $q$ and actual inflation $I$ as a proportion of actual inflation:
%
\begin{equation}
    E_{q} = \frac{F_{q} - I_{q}}{I_{q}}.
\end{equation}

This is different from the accuracy measure \cite{Frendreis2000} used in their preliminary examination of forecast errors. They averaged the absolute value of yearly inflation forecast errors over a 19 year period\footnote{i.e. $\frac{|F_{y} - I_{y}|}{19}$} to examine Federal Reserve accuracy. Their measure has a number of drawbacks. First, it does not give us any indication of the direction of the forecast error, which is crucial for examining possible partisan biases. In their comparison of CBO and administrations' forecasts they did use a simple dichotomous directional indicator of accuracy in a given year (i.e. a forecast greater than or less than the actual level). This does not give us a sense of the relative size of the errors and could easily amplify trivial results. Almost any forecast will be above or below the actual inflation level in all but the unusual cases where the forecasts exactly equal the actual inflation level.

Second, the average of the absolute errors values could be highly skewed by years of unusually large errors, which is more likely in years of higher inflation. This is not a trivial concern because the inflation level varies substantially over time (see Figure \ref{absolute}).\footnote{\cite{Frendreis2000} also do not include any other indication of the errors' distribution.} So, we choose to focus on proportional rather than absolute errors by quarter to avoid focusing on a parameter that is highly vulnerable to absolute value outliers. Quarterly proportional errors are also more substantively meaningful for comparing errors across time periods.\footnote{Note that the direction and significance of our main findings do not change when we use absolute rather than proportional errors in our estimation models (discussed below). The magnitude does change, but this is to be expected because the range of the absolute inflation errors is much larger than proportional errors. These results are available from the authors upon request.}

Third, using multi-year or even year-level indicators makes it difficult to examine biases in the run up to an election or any other process that may be observed through variations within years. Using quarterly data--the smallest level available--gives us a much more detailed view of any processes that might influence accuracy.

If the forecasts are unbiased the mean error of the forecasts--using either \cite{Frendreis2000} or our measure--would be indistinguishable from zero. While \cite{Frendreis2000} found that Fed errors were low relative to presidential administrations' on average over a 19 year period and \cite{Romer2000} found that the Fed's internal forecasts meet the requirement for unbiasedness on average over the full history of Greenbook forecasts, such amalgamations disguise long periods of over- or under-predictions of inflation, as noted in \cite{Capistran2006} and illustrated in Figure \ref{absolute}. Within economics the Fed's forecasts have been examined for evidence of rationality. These studies generally find that the Fed rationally incorporates information into its forecasts, outperforming private forecasts \cite[c.f.][]{Gamber2009}. These studies, however, have rarely incorporated Fed Staff member' political beliefs or preferences, because Federal Reserve Staffers are assumed to be politically independent.

Our dataset has 169 forecast quarters,\footnote{This is the maximum number of observations. Longer forecasts result in fewer forecasted quarters. Likewise, some forecast lengths are unavailable for the full time period.} spanning the fourth quarter of 1965 through the end of 2007. Greenbook forecasts correspond to those provided for the FOMC meeting closest to the middle of the quarter. We found actual inflation corresponding to each of these quarters\footnote{Inflation was calculated by comparing quarters year-on-year. The exact inflation measure that the Federal Reserve was forecasting changed a number of times, so the measure of actual inflation used to created the forecast error variable changes accordingly. The GNP deflator indicator is used from the beginning of our sample through the end of 1991. From the first quarter of 1992 through the first quarter of 1996 actual inflation is measured with the GDP deflator. From the second quarter of 1996 we use the chain-weighted GDP price index. For more details on how the forecasted quantity changed see the Greenbook data description file available at: \url{http://www.phil.frb.org/research-and-data/real-time-center/greenbook-data/philadelphia-data-set.cfm}. The Greenbook inflation forecast variable we used is called ``PGDPdot''.} using data from the Federal Reserve's FRED website.\footnote{See \url{http://research.stlouisfed.org/fred2/}. Accessed December 2011.} We examine errors made by forecasters in the current quarter and all quarters up to five quarters before.\footnote{The Greenbook contains very incomplete data for forecasts made over longer time spans.} The results are generally the same regardless of the forecast's age, e.g. the results were similar for predictions made $q - 1$ quarters before the forecasted quarter $q$, $q - 2$ quarters before, and so on. In particular the presidential partisan findings are robust regardless of forecast age (see Figure \ref{ExpectValueParty}). For simplicity, the majority of results we show and discuss in detail are from models with forecasts made two quarters beforehand.\footnote{Using these two quarter forecasts restricts our observations because they are rarely available before the 1970s.} Figure \ref{absolute} compares absolute actual inflation for each quarter and inflation forecasts made two quarters before.

%%%%%%%%%%%%%%%%%%%%%%%%%%  Raw Greenbook estimate vs. actual graph
\begin{figure}[t]
    \caption{Greenbook Inflation Forecasts Made 2 Qtr. Beforehand and Actual Quarterly Inflation}
    \label{absolute}
    \begin{center}

\begin{knitrout}
\definecolor{shadecolor}{rgb}{0.969, 0.969, 0.969}\color{fgcolor}\begin{kframe}


{\ttfamily\noindent\bfseries\color{errorcolor}{\#\# Error: cannot change working directory}}\end{kframe}
\end{knitrout}

    \end{center}
    \begin{singlespace}
        {\scriptsize{The vertical grey dotted line indicates when the Federal Reserve Board Global (FRB/Global) forecasting model was fully implemented.
                      }}
    \end{singlespace}
\end{figure}


\subsection{Are There Partisan Forecast Errors?}

\begin{knitrout}
\definecolor{shadecolor}{rgb}{0.969, 0.969, 0.969}\color{fgcolor}\begin{kframe}


{\ttfamily\noindent\bfseries\color{errorcolor}{\#\# Error: object 'cpi.data' not found}}\end{kframe}
\end{knitrout}



















