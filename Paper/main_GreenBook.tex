%%%%%%%%%%%%%%%%%%%%%%%%%%%%
% Do Elections Affect Fed Forecasts?
% Cassandra Grafström and Christopher Gandrud
%%%%%%%%%%%%%%%%%%%%%%%%%%%%

% !Rnw weave = knitr

\documentclass[a4paper]{article}
\usepackage{fullpage}
\usepackage[authoryear]{natbib}
\usepackage{setspace}
    \doublespacing
\usepackage{hyperref}
\hypersetup{
    colorlinks,
    citecolor=black,
    filecolor=black,
    linkcolor=cyan,
    urlcolor=cyan
}
\usepackage{dcolumn}
\usepackage{booktabs}
\usepackage{url}
\usepackage{tikz}
\usepackage[utf8]{inputenc} 

%%%%%%% Title Page %%%%%%%%%%%%%%%%%%%%%%%%%%%%%%%%%%%%%%%%%%%%
\title{Does Partisanship Affect Fed Inflation Forecasts?}

\author{Christopher Gandrud and Cassandra Grafstr\"{o}m}

\usepackage{Sweave}
\begin{document}

\Sconcordance{concordance:main_GreenBook.tex:main_GreenBook.Rnw:%
1 30 1 54 0 1 3 76 1 1 0 1 1 83 0 36 1 1 0 14 1 4 0 14 1 8 0 5 1 1 %
0 69 1 1 8 12 1 1 2 21 1 7 0 3 1 1 2 29 1 4 0 9 1 1 6 31 1 1 4 31 %
1 4 0 17 1 46 0 34 1 33 0 28 1 38 0 21 1 1 5 18 1 3 0 15 1 1 17 13 %
1 8 0 12 1 8 0 12 1 1 4 17 1}


\maketitle

%%%%%%% Abstract %%%%%%%%%%%%%%%%%%%%%%%%%%%%%%%%%%%%%%%%%%%%
\begin{abstract}
\noindent\emph{Very Early draft version. Comments welcome.}\footnote{The paper is written with {\tt{knitr}} \citep{knitr2012} and is fully reproducible. Please contact us for the replication files.} \\[0.2cm]
Recent work argues that the Federal Reserve is not politically indifferent \citep{Clark2011}. The Fed tends to choose looser monetary policies during Republican administrations, possibly in order to ensure the (re)election of ideologically preferred administrations. This model excludes an essential aspect of monetary policy decision-making: expectations about future inflation. We use the Fed's Green Book forecasts and presidential election polling data to test whether expected electoral outcomes shape the estimates of future economic performance that influence FOMC policies. We find that Federal Reserve staff probably do not bias their forecasts to influence Fed governors around elections. However, they do systematically overestimate inflation during Democratic presidencies and underestimate inflation during Republican ones. This suggests that while not electorally motivated, Fed staff have a partisan bias when making inflation forecasts. Samson the puppy.

\end{abstract}

\begin{description}
  \item [{\textbf{Keywords:}}] forecast bias, Federal Reserve, rational partisan cycle,
\end{description}

\vspace{0.3cm}

Recent work argues that the Federal Reserve is not politically indifferent \citep{Clark2011}. The Fed tends to choose looser monetary policies during Republican administrations, possibly in order to ensure the (re)election of ideologically preferred administrations. This bias is assumed by Clark and Arel-Bundock to arise from a Board of Governors that prefer rightist presidents to leftist ones and so set interest rates to help Republican incumbents and hinder Democratic ones. 

Does this partisan preference also affect Federal Reserve staff's forecasts, which in turn influences Governors' interest rate decisions? 

An alternative possibility is that Federal Reserve staff may believe that inflation will be much higher under Democratic presidents. Employees within the Federal Reserve Banks may expect that Democratic presidents produce policies that increase inflation and Republican presidents produce policies that limit inflation. However, Republican and Democratic administrations both engage in largely expansionary fiscal policies. We argue that this leads to large overestimations of inflation during Democratic presidencies and a significant underestimation of inflation during Republican ones.

%THOUGHTS: THE DIFFERENCE IN ERRORS IMPLIES SOME SORT OF NAIVITE ON THE PART OF THE FED FORECASTERS IF THEY ALWAY THINK THAT THE REPUBLICAN PRESIDENT WILL SIGNIFICANTLY REDUCE SPENDING GROWTH AND FAILS TO WHILE THE DEMOCRATIC PRESIDENTS WILL SIGNIFICANTLY INCREASE SPENDING GROWTH AND FAIL TO. MAYBE THEY THINK THAT THE REPUBLICANS WILL ENGAGE IN POLICIES THAT LEAD TO HIGHER REAL GROWTH THROUGH THEIR TAX CUTS WHILE THE DEMOCRATS WILL ENGAGE IN PURELY SPENDING INCREASES (OR WITH TAX INCREASES) 

%CHRISTOPHER: HAVE WE TESTED TO SEE IF INFLATION IS ACTUALLY HIGHER UNDER DEMOCRATIC PRESIDENTS (ESPECIALLY AT THE ENDS OF THEIR TERMS) THAN IT IS UNDER REPUBLICAN PRESIDENTS? SINCE I BASICALLY HAVEN'T TOUCHED THE DATA I CAN'T REMEMBER. IF IT IS THEN THIS WOULD BE A NICE THING TO KNOW. IF THEY ARE ABOUT THE SAME THEN WE ARE BASICALLY SAYING THAT ECONOMISTS IN THE FED ARE NOT UPDATING (AND THIS IS IN LINE WITH BILL AND VINCENT'S ARGUMENT), WHEREAS IF THEY DIFFER THEN THEY ARE SIMPLY OVERSHOOTING BUT AREN'T NECESSARILY WRONG ON THE BASIC PREMISE.

In this paper we first provide a brief discussion of what Green Book inflation forecasts and forecast errors are, including their importance for monetary policy making and our current understanding of how they are made. As we demonstrate in this first section, academic scholarship up till now has not examined possible partisan causes of forecast erros. We then introduce the issues of inflation forecast partisan biases and posit a number of ways that Green Book forecasting may be influenced by them. We test these theories with a series of regression models using both unmatched and matched data on Green Book inflation forecast errors from the 1970s through 2005. The models suggest that, even when controlling for a number of important economic and political factors, Green Book forecasts show a distinct presidential partisan bias. 

%%%%%%%%%%%%%% Section 1: Forecasting Inflation at the Fed %%%%%%%%%%%%%%%%%%%%%

\section{Forecasting Inflation at the Fed \& Inflation Forecast Errors}

In this section we briefly describe what Green Book inflation forecasts and forecast errors are, why they are an important part of monetary policymaking in the United States, and the current understanding of how Federal Reserve inflation forecasts were made since the late 1960s.

\subsection{Forecasting \& Forecasting Errors}

Federal Reserve staff create ``Green Book" forecasts\footnote{Green Book data can be found at {\url{http://www.phil.frb.org/research-and-data/real-time-center/greenbook-data/philadelphia-data-set.cfm}}. Accessed December 2011.} before each meeting of the Federal Open Markets Committee (FOMC), the Federal Reserve's primary decision-making body. We focus on GNP/GDP price index forecasts.\footnote{Note: GNP was used to 1991 (inclusive) and GDP was used from 1992. Furthermore, the implicit deflator was used before the second quarter of 1996 and chain-weighted price index was used from the second quarter of 1996 onwards.} Green Book forecasts are made available for each quarter from the fourth quarter of 1964 through the end of 2005\footnote{There is a five year lagged release schedule}. We have 161 forecast quarters in our data set. Forecasts correspond to the FOMC meeting closest to the middle of the quarter. For a given quarter the data includes forecasts made in the present quarter and up to 5 quarters before. Actual inflation corresponding to each of these quarters\footnote{Inflation was calculated by comparing quarters year-on-year.} was found using data from the Federal Reserve Economic Data website.\footnote{\url{http://research.stlouisfed.org/fred2/}} Indicators comparable to the forecasted quantity are used, e.g. from the second quarter of 1996 we use the chain-weighted GDP price index. Absolute actual inflation for each quarter and inflation forecasts made two quarters before are compared in Figure \ref{absolute}. In general we use forecasts made two quarters before.\footnote{Using these two quarter forecasts constricts our observations so to 150 since, apart from the first quater of 1968, they are not included in the Green Book data before the fourth quarter of 1968.} 

%%%%%%%%%%%%%%%%%%%%%%%%%%  Raw Green Book estimate vs. actual graph
\begin{figure}[t]
    \caption{Green Book Inflation Forecasts and Actual Quarterly Inflation}
    \label{absolute}
    \begin{center}
    
