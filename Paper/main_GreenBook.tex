%%%%%%%%%%%%%%%%%%%%%%%%%%%
% Do Elections Affect Fed Forecasts?
% Cassandra Grafström and Christopher Gandrud
% 18 January 2013
%%%%%%%%%%%%%%%%%%%%%%%%%%%%

% !Rnw weave = knitr

\documentclass[a4paper]{article}
\usepackage{fullpage}
\usepackage[authoryear]{natbib}
\usepackage{setspace}
    \doublespacing
\usepackage[usenames,dvipsnames]{xcolor}
\usepackage{hyperref}
\hypersetup{
    colorlinks,
    citecolor=black,
    filecolor=black,
    linkcolor=cyan,
    urlcolor=cyan
}
\usepackage{dcolumn}
\usepackage{booktabs}
\usepackage{url}
\usepackage{tikz}
\usepackage{todonotes}
\usepackage[utf8]{inputenc} 

% Set knitr global options



%%%%%%% Title Page %%%%%%%%%%%%%%%%%%%%%%%%%%%%%%%%%%%%%%%%%%%%
\title{Does Presidential Partisanship Affect Fed Inflation Forecasts?}

\author{Christopher Gandrud \\
                {\emph{Yonsei University}}\footnote{\href{mailto:gandrud@yonsei.ac.kr}{gandrud@yonsei.ac.kr}} \\
                and \\
            Cassandra Grafstr\"{o}m \\
                {\emph{Hertie School of Governance \& University of Michigan}}\footnote{\href{mailto:cgrafstr@umich.edu}{cgrafstr@umich.edu}}}

\begin{document}

\maketitle

%%%%%%% Abstract %%%%%%%%%%%%%%%%%%%%%%%%%%%%%%%%%%%%%%%%%%%%
\begin{abstract}
\noindent\emph{Working draft. Comments welcome.}\footnote{Thank you to the Mark Hallerberg and the Fiscal Governance Centre at the Hertie School of Governance for comments and support. Thank you also to Vincent Arel-Bundock, Thomas Stark, and seminar participants at the Hertie School of Governance, London School of Economics and Yonsei University. \\
This paper was written using {\tt{knitr}} \citep{knitr2013}. It can be entirely replicated from data, analysis source code, and markup files available on our GitHub page at: {\url{https://github.com/christophergandrud/GreenBook}}.} \\[0.2cm]

Recent work argues that policy-makers at the United States Federal Reserve are not politically indifferent \citep{Clark2012}. The Fed tends to loosen monetary policies during Republican administrations, possibly in order to ensure the (re)election of ideologically preferred presidents, while tightening them throughout Democratic ones. However, expectations about future inflation ought to be driving monetary policy. Are these findings really driven by different assumptions about the (inflationary) policies of the different parties? We use the Fed's Greenbook forecasts to test whether presidents' partisan identification shapes the estimates of future economic performance that influence FOMC policies. We find that Federal Reserve staff probably {\emph{do not}} bias their forecasts to influence Fed governors around elections. However, they {\emph{do}} systematically overestimate inflation during the entire tenure of Democratic presidencies and underestimate inflation during Republican ones. This suggests that while not electorally motivated, Fed staff inflation forecasts have a presidential partisan bias.

\end{abstract}

\begin{description}
  \item [{\textbf{Keywords:}}] forecast bias, Federal Reserve, rational partisan cycle, heuristics, interest rate, inflation, monetary policy
\end{description}

\vspace{0.3cm}

Monetary policy is an inherently forward looking enterprise. Beliefs about the economy's future course ought to guide the setting of interest rates. In the United States the Federal Open Market Committee (FOMC) makes monetary policy based on forecasts provided to them by Federal Reserve \emph{staff}. A significant determinant of changes in growth, inflation, and unemployment are the policies of the federal government. Expectations about deficit spending, labor and financial market regulations, and myriad other policies affect the Federal Reserve's (Fed) forecasts. The party to which the president belongs can serve as a cue for the types policies likely to be pursued during his administration: Republicans being expected to pursue more low inflation policies and Democrats being expected to pursue more low unemployment, but more inflationary, policies. Evidence suggest, however, that the real differences in policies implemented by the two parties is quite minimal \citep{Bartels2008}. How does the US Federal Reserve incorporate the government's partisan composition when forming expectations of future inflation? If Fed staff expect higher inflation under Democrats than Republicans, these beliefs could be incorporated into the forecasts used by the Board of Governors to make monetary policies.

We provide evidence that Fed internal inflation forecasts consistently predict that inflation will be lower than it turns out to be under Republican presidencies and predict that inflation will be higher than it turns out to be under Democratic administrations. Even accounting for changes in monetary policy and a variety of other economic and political factors, Federal Reserve economists over-shoot inflation forecasts for Democrats and under-shoot for Republicans. We find evidence that partisan heuristics of policy expectations explain these predictive failures.

In this paper we first provide a brief discussion of what inflation forecasts, especially so called ``Greenbook" forecasts, and forecast errors are, including their importance for monetary policy-making and our current understanding of how they are made. As we demonstrate in this first section, academic scholarship up until now has not examined possible partisan causes of forecast errors. We then introduce the issue of partisan biases in inflation forecasts, expounding three different ways that presidential partisanship could affect Fed staff expectations, and posit how forecasting may be influenced by each of them. We test these theories with a series of regression models using both unmatched and matched data on Greenbook inflation forecast errors from the 1970s through 2006. Our findings suggest that even when controlling for a number of important economic, bureaucratic and political factors, Greenbook forecasts show a distinct presidential partisan bias across presidential terms. In the conclusion we discuss the implications of these findings for monetary policy and election outcomes and future directions for research.

%%%%%%%%%%%%%% Section 1: Forecasting Inflation at the Fed %%%%%%%%%%%%%%%%%%%%%

\section{Forecasting Inflation at the Fed}

In this section we briefly describe what Greenbook inflation forecasts and forecast errors are, why they are an important part of monetary policy-making in the United States, and describe how Federal Reserve inflation forecasts have been made since the late 1960s.

\subsection{Forecasting \& Forecasting Errors}

Prior to every FOMC meeting Federal Reserve staff create a document called the ``Current Economic and Financial Conditions" or ``Greenbook" that contains information on recent behavior and forecasts of various macroeconomic aggregates.\footnote{Greenbook data can be found at {\url{http://www.phil.frb.org/research-and-data/real-time-center/greenbook-data/philadelphia-data-set.cfm}}. Accessed October 2012.} Federal Reserve staff make forecasts of various elements of the US and global economies so that the FOMC can make policies appropriate to fulfill the Fed's dual mandate of maintaining maximum employment and price stability. Greenbook forecasts are currently available to the public for each quarter from the fourth quarter of 1965 through the end of 2006.\footnote{There is a five year lagged release schedule. Also, some forecasts are not available for the entire period.}  We focus on GNP/GDP price index forecasts in our study.\footnote{Note: the exact index forecasted changed over time. GNP was used through 1991 (inclusive) and GDP was used from 1992 onward. Furthermore, the implicit deflator was used before the second quarter of 1996 and chain-weighted price index was used from the second quarter of 1996 on-wards. We made sure to standardize the inflation measures in our data set to match these changes. For more details on how the forecasted index changed see the Greenbook data description file available at: \url{http://www.phil.frb.org/research-and-data/real-time-center/greenbook-data/philadelphia-data-set.cfm}. The Greenbook variable we used is called ``PGDPdot''.} We choose this indicator of Federal Reserve forecast accuracy both because there is a strong theoretical assumption that central bankers are more inflation averse than are politicians \citep{Cukierman1992,Mukherjee2008,Tillmann2008} and because this is the dominant measure of forecast errors used in the economics literature \citep[c.f.][]{Romer2000}. The FOMC uses these forecasts to determine the appropriate monetary policy to pursue. Higher inflationary expectations increase the likelihood of the FOMC tightening policy in order to slow inflation and an overheating economy; low inflationary expectations increase the likelihood of a loosening of monetary policy to bolster growth and employment, all else equal.

It is important to note that finalized forecasts of macroeconomic aggregates are a combination of both econometric models and the professional opinions of forecasters about likely changes in the economy's trajectory not necessarily picked up in these models \citep{Karamouzis1989,Reifschneider1997}. The forecasts contained in the Greenbook are the ``consensus" forecasts combining {\bf{both}} econometric models and judgmental forecasts assuming current monetary policies remain unchanged. Greenbook forecasts thus do not provide us with any information on what the econometric forecasts were prior to being integrated into the ``consensus" forecast presented to the FOMC. Below we briefly explain how the econometric aspects of the consensus forecasts have been modeled.

\subsection{Our Current Understanding of Fed Inflation Forecasting}

Federal Reserve staff have used two primary sets of econometric models during the period for which Greenbook data is available.\footnote{This subsection draws heavily on Brayton et al.'s \cite{Brayton1997} detailed description of the changes to Federal Reserve forecasting models that took place in 1996.} This section describes these two models and their importance for our understanding of forecast errors. It is important to reiterate that finalized forecasts of macroeconomic aggregates are a combination of both these mathematical models and the professional opinions of forecasters about likely changes in the economy's trajectory not necessarily picked up in the models \citep{Karamouzis1989,Reifschneider1997,Taylor1997}. The most significant change was the move to the so-called Federal Reserve Board (FRB) Models in the mid-1990s that incorporated rational, as opposed to adaptive, expectations by market actors. We discuss the early and current models. In the following section we discuss the implications each has for forecast errors.

\paragraph{Early models of the economy}
The first simultaneous equation model of the US economy was developed and adopted by the Federal Reserve between 1966 and 1975. The model, originally developed in conjunction with MIT, University of Pittsburgh, and the Social Science Research Council (MPS), was based on a neo-classical growth model of production and factor demands and embraced the IS/LM/Phillip's Curve paradigm. This model was composed of about 125 equations modeling various interdependent aspects of the American economy in a simultaneous equations framework. Homogeneity assumptions ensured neutrality of money in the long-run--that is, changes in the stock of money only affects nominal variables in the economy such as wages or prices but these changes have no effect on real (inflation adjusted) variables like real GDP or employment in the long-run because any increase in the money supply will be offset by a proportional increase in prices and wages eventually. The essential feature of this model, however, was that the public held adaptive expectations. Basing a model on adaptive expectations means that people's expectations about what will happen in the future is based on what happened in the past. These models simply extrapolated future behavior of the economy from its recent past behavior, taking no account of the fact that people may change their behavior in response to future expected outcomes.

The collapse of Bretton Woods spurred a number of changes to the model. First, following the introduction of floating exchange rates, the trade and exchange rate sections of the domestic economy part of the model were expanded. Second, and more significantly, an explicit model of the global economy was developed. The Multi-Country Model (MCM) introduced in 1975 originally included estimates of macroeconomic performance in the US, Canada, Germany, Japan, the UK, and ``the rest of the world sector." This model was again based on the short-run dynamics of of the IS/LM/Phillip's Curve and long-run neo-classical growth model.

Both the MPS and MCM models were tweaked during the 1980s, with about one-third of the equations in the MPS model changed during this time. The MCM was also expanded to include a larger set of major trading partners.\footnote{The post-1992 model included each of the G-7 countries individually as well as Mexico, and blocks representing the OECD, newly industrialized countries, OPEC and the rest of the world.} However, the basic assumptions of the models, specifically the adaptive expectations assumptions, remained unchanged. The exclusive use of VAR models (solely backwards looking actors) meant that the models failed to explicitly account for actors concerns about future outcomes. The rational expectations revolution in economics in the 1970s and 1980s made this assumption an increasingly controversial one. Thus, the development of new models began in earnest in 1991.
 
\paragraph{Current models of the economy}
New models of the American economy's near-term trajectory were used in the 1990s. The Federal Reserve Board US model (FRB/US) is composed of approximately 40 behavioral equations, estimated with single-equation techniques. This model explicitly considers the role of economic expectations in economic behavior. The foundational assumption of adaptive expectations in the old model was thus replaced with rational or model-consistent expectations. In these models, prices are sticky and aggregate demand determines short-run output. Further, monetary policy's effects on the economy are extensively modeled. 

The Federal Reserve Board Global (FRB/Global) model's development began in 1993 and had replaced the old models by 1996. The FRB/Global model links the behavioral equations of FRB/US with approximately 200 behavioral equations representing the other 11 regions of the model. Anticipated values of future variables directly influence interest and exchange rates, components of aggregate demand, wages and prices.

\subsection{Forecast Accuracy}\label{ForecastAcc}

How accurate are Fed staff forecasts? We measure accuracy by calculating {\bf{forecast error}} $E$ as the difference between the Greenbook inflation forecast $F$ for a given quarter $q$ and actual inflation $I$ as a proportion of actual inflation
%
\begin{equation}
    E_{q} = \frac{F_{q} - I_{q}}{I_{q}}
\end{equation}
%
We choose to focus on proportional rather than absolute error\footnote{i.e. $F_{q} - I_{q}$} because it is substantively more meaningful for comparing errors across time periods. As we saw in Figure \ref{absolute}, absolute inflation varies considerably across different periods. So, to be able to estimate meaningful effects on errors across time we need look at proportional rather than absolute error.\footnote{Note that the direction and significance of our main findings do not change when we use absolute rather than proportional errors in our models. The magnitude does change, but this is to be expected because the range of the absolute inflation errors is much larger than proportional errors. These results are available from the authors upon request.}

If the forecasts are unbiased then the mean error of the forecasts would be indistinguishable from zero. While \cite{Romer2000} found that the Federal Reserve's internal forecasts  meet this requirement over the full history of Greenbook forecasts, such an amalgamation disguises long periods of over- or under-predicting inflation, as noted in \cite{Capistran2006} and illustrated in Figure \ref{absolute}. Within economics the Fed's forecasts have been examined for evidence of rationality. These studies generally find that the Fed rationally incorporates information into its forecasts, outperforming private forecasts \cite[c.f.][]{Gamber2009}. These studies, however, have rarely incorporated political preferences of the Fed because Federal Reserve staffers are assumed to be politically independent.

We have 165 forecast quarters in our data set, spanning the fourth quarter of 1965 through the end of 2006. Forecasts correspond to the FOMC meeting closest to the middle of the quarter. We found actual inflation corresponding to each of these quarters\footnote{Inflation was calculated by comparing quarters year-on-year. The exact inflation measure that the Federal Reserve was forecasting changed a number of times, so the measure of actual inflation used to created the forecast error variable changes accordingly. The GNP deflator indicator is used from the beginning of our sample through the end of 1991. From the first quarter of 1992 through the first quarter of 1996 actual inflation is measured with the GDP deflator.  From the second quarter of 1996 we use the chain-weighted GDP price index.} using data from the Federal Reserve's FRED website.\footnote{See \url{http://research.stlouisfed.org/fred2/}. Accessed December 2011.} We examine errors made by forecasters in the current quarter and all quarters up to five quarters before.\footnote{The Greenbook contains very incomplete data for forecasts made over longer time spans.} The results are generally the same regardless of the forecast used, especially the presidential partisan findings (see Figure \ref{ExpectValueParty}). For simplicity, the majority of results we show and discuss in detail are from models with forecasts made two quarters beforehand.\footnote{Using these two quarter forecasts restricts our observations because they are rarely available before the 1970s.} Figure \ref{absolute} compares absolute actual inflation for each quarter and inflation forecasts made two quarters before.

%%%%%%%%%%%%%%%%%%%%%%%%%%  Raw Greenbook estimate vs. actual graph
\begin{figure}[t]
    \caption{Greenbook Inflation Forecasts Made 2 Qtr. Beforehand and Actual Quarterly Inflation}
    \label{absolute}
    \begin{center}
    
\begin{knitrout}
\definecolor{shadecolor}{rgb}{0.969, 0.969, 0.969}\color{fgcolor}

{\centering \includegraphics[width=0.8\linewidth]{figure/BaseInflation} 

}


\end{knitrout}

    
    \end{center}
    \begin{singlespace}
        {\scriptsize{Forecasts were made two quarters beforehand. \\
                     The grey dotted lines indicate the {\emph{approximate}} years that the Simultaneous Equation Models (SEM) and Federal Reserve Board Global (FRB/Global) forecasting models were fully implemented.  
                      }}
    \end{singlespace}
\end{figure}


\subsection{Are There Partisan Forecast Errors?}

\begin{knitrout}
\definecolor{shadecolor}{rgb}{0.969, 0.969, 0.969}\color{fgcolor}\begin{kframe}
\begin{flushleft}\ttfamily\noindent\bfseries\textcolor{errorcolor}{\#\# Error: there is no package called 'plotrix'}\end{flushleft}\end{kframe}
\end{knitrout}


Forecasts should be unbiased in that they have a mean error of zero \citep[5]{Bruck2006}. Using this criteria, forecast errors should be the same--ideally with a mean of 0--regardless of the incumbent president's party identification. This is not the case. From the second quarter of 1969 through 2006 the mean standardized forecast error was -0.03, i.e. forecasts were off by about -3 percent. The mean error is noticably different across Republican compared to Democratic presidencies. Across Republican presidencies it was \begin{flushleft}\ttfamily\noindent\bfseries\textcolor{errorcolor}{ \\ 
Error in eval(expr, envir, enclos) : object 'RepMean' not found}\end{flushleft} percent and \begin{flushleft}\ttfamily\noindent\bfseries\textcolor{errorcolor}{ \\ 
Error in eval(expr, envir, enclos) : object 'DemMean' not found}\end{flushleft} percent across Democratic presidencies.\footnote{These means are from estimates made two quarters beforehand. Both means are statistically significantly different from 0, the full observation mean, and each other at the 99\% confidence level. For more details see \url{http://bit.ly/WHsRYh}.} On average, inflation was underestimated in Republican presidencies and overestimated in Democratic ones.

Figure \ref{errors_over_time} plots forecast errors across our sample separated by presidential term and party. The first thing to note is that inflation was rarely underestimated during the three Democratic presidential terms in our sample. The underestimates that were made were relatively small. Also, the largest overestimates we see were made during Bill Clinton's (Democratic) presidency. All of the major inflation underestimates were made during Republican presidencies, particularly during Richard Nixon's, Gerald Ford's, and George W. Bush's presidencies. Inflation was often overestimated during the second part of Reagan's first term, his second term and George H.W. Bush's term. Over this period--often referred to as the Volcker Revolution \citep[see][]{Bartels1985}--inflation was suddenly much lower than before (see Figure \ref{absolute}). It may have taken awhile for predictions to adjust to this new lower level of inflation, particularly because the Fed's own models of the economy assumed that money had no real effects on the economy during this period (see above).

This summary examination of inflation forecast errors suggests that there may be a partisan bias. What might explain this?

%%%%%%%%%%%%%%%%%%%%%%%%%%   Greenbook Error Across Time
\begin{figure}[t]
    \caption{Errors in Inflation Forecasts Made 2 Qtr. Beforehand (1969 - 2006)}
    \label{errors_over_time}
    \begin{center}
    
\begin{knitrout}
\definecolor{shadecolor}{rgb}{0.969, 0.969, 0.969}\color{fgcolor}

{\centering \includegraphics[width=0.8\linewidth]{figure/PartisanError} 

}


\end{knitrout}

    
    \end{center}
    \begin{singlespace}
        {\scriptsize{Note: An error of 0 indicates that inflation was perfectly predicted.}}
    \end{singlespace}
\end{figure}


%%%%%%%%%%%%%%%%%%%%%%%%%%%%% Section 2: Possible Explanations of Fed Inflation Forecast (In)accuracy %%%%%%%%%
\section{Possible Explanations of Fed Inflation Forecast Inaccuracy}

In this section we try to understand trends in Fed inflation forecast errors by discussing theories derived from the the political economy literature that provide possible explanations of why forecasts seem to have a partisan bias. 

\subsection{Econometric Models \& Accuracy}

Before digging into the partisan explanations of forecast errors, which would largely be the result of Federal Reserve staff judgement, it is worth hypothesizing that perhaps forecast inaccuracy is the result of systematic errors in the Staff's predictive econometric models. Presumably, the move to rational expectations ought to improve forecast accuracy relative to the earlier period. The goal of incorporating forward looking actors into the models was to account for an important source of endogeneity in the earlier models that could lead to overestimates of important economic indicators under some circumstances and underestimates of those same indicators under others. None of these over or underestimates, however, ought to have been linked to the party of the president. We would, however, expect that the magnitude of forecast errors' changed after 1996.

%Random thought: what if the models assume that voters have partisan expectations and behave according to those partisan characteristics. Then when the partisans fail to do so or when voters react to these as they should (by not reacting) this results in over or under-shooting inflation by the Fed, particularly under the early model?

% That's a good thought. It is a bit of a tweak on our argument, but not a fundamental challenge. To give it weight we need to talk to Fed staff I think.

\subsection{Partisan Biases \& Accuracy}

There is an extensive literature looking for partisan effects in monetary policy, but no studies examining the impact of presidential partisanship on predictions of inflation. Studies have looked for evidence of partisan preferences manifesting themselves in the monetary policies chosen by the FOMC. There are two essential reasons that partisan effects would be found in policy: either a preference for one party over another by members of the FOMC or an expectation that the parties will engage in systematically different policies that will influence inflation leading the FOMC to support more preferred policies and attempt to inhibit less preferred ones. 

The preference arguments assume that a conservative central banker will prefer the election of politicians who hold more similar inflationary preferences (i.e., those with a stronger preference for low inflation) and enact policies to bolster their preferred candidate's prospects of being elected. In the US this would mean that the Federal Reserve would implement policies that supported the electoral prospects of Republican incumbents and harm the electoral prospects of Democratic incumbents \citep{Clark2012,Hakes1988,Sieg1997,Tootell1996}.

The rational partisan expectations literature assumes that central bankers do not have an innate preference for one party over another, but instead expect Democrats and Republicans to behave differently in office \citep{Alesina1991,Hibbs1994}. It is these behavioral expectations that would lead to different monetary policy outcomes under Democratic and Republican presidencies, with the former expected to engage in more expansionary and inflationary policies than the latter. In order to stave off a rise in inflation under a Democrat the Fed would tighten monetary policy (raise interest rates or reduce the money supply); because Republicans are expected to prefer low inflation, they will pursue low inflation policies and so the FOMC can accommodate Republican presidents' policies without fear of stoking inflation. This argument is again based on the assumed preferences of partisans, but does not require the FOMC to be politically biased as the former does. 

Arguments similar to those in the literature on partisan monetary cycles can also be applied to predicting differences in inflation forecasts. We present three theoretical arguments about how inflation expectations would differ by the President's party: the partisan preference theory, the monetary expectations theory, and the partisan heuristics theory (see Figure \ref{ExpectGraphs} for an illustration of the theories' contrasting inflation error predictions). Of the three, we find the partisan heuristics theory to be most consistent with the data.




% Define colors for figure
%% See: http://colorbrewer2.org/
\definecolor{DEM}{HTML}{2259B3}
\definecolor{REP}{HTML}{C42B00}

\begin{figure}
  \caption{Stylized Inflation Forecast Error Predictions}
  \label{ExpectGraphs}
  \begin{center}

    \vspace{0.25cm}

    \tikzstyle{bagMain} = [text width = 5cm]
    \tikzstyle{bagDem} = [text = DEM]
    \tikzstyle{bagRep} = [text = REP]
    
    \tikzstyle{DemLine} = [draw, 
                          color=DEM,
                          opacity=0.9,
                          line width=1.5mm]

    \tikzstyle{RepLine} = [draw, 
                          color=REP,
                          opacity=0.9,
                          line width=1.5mm]

\begin{tikzpicture}

  %%%% Partisan Preferences
  \node (PP) at (-7, 5) [bagMain]{{\bf{Partisan Preferences}}};
  \node (E) at (-10.5, 3.25) [bagMain, rotate=90]{{\emph{Forecast Error}}};
  \node (T) at (-7.3, -0.5) [bagMain]{{\emph{Duration of Pres. Term}}};
  
  \draw (-10, 0) -- (-6, 0);
  \draw (-10, 0) -- (-10, 4);
  
  \draw[RepLine] (-9.5, 1.9) -- (-7.5, 1.9); 
  \draw[DemLine] (-9.5, 2.1) -- (-7.5, 2.1); 

  \draw[RepLine] (-7.52, 1.9) -- (-6.5, 1); 
  \draw[DemLine] (-7.52, 2.1) -- (-6.5, 3); 

  \node (R1) at (-6.5, 0.5) [bagRep]{Rep.};
  \node (D1) at (-6.5, 3.5) [bagDem]{Dem.};

  
  %%%% Monetary Expectations
  \node (PP) at (-2, 5) [bagMain]{{\bf{Monetary Expectations}}};
  
  \draw (-5, 0) -- (-1, 0);
  \draw (-5, 0) -- (-5, 4);
  
  \draw[RepLine] (-4.5, 2.1) -- (-2.5, 2.1); 
  \draw[DemLine] (-4.5, 1.9) -- (-2.5, 1.9); 

  \draw[DemLine] (-2.52, 1.9) -- (-1.5, 1); 
  \draw[RepLine] (-2.52, 2.1) -- (-1.5, 3); 

  \node (D2) at (-1.5, 0.5) [bagDem]{Dem.};
  \node (R2) at (-1.5, 3.5) [bagRep]{Rep.};
  
  
  %%%% Partisan Heuristics
  
  \node (PP) at (3, 5) [bagMain]{{\bf{Partisan Heuristics}}};
  
  \draw (0, 0) -- (4, 0);
  \draw (0, 0) -- (0, 4);
  
  \draw[DemLine] (0.5, 3) -- (4, 3); 
  \draw[RepLine] (0.5, 1) -- (4, 1); 

  \node (D3) at (3.5, 3.5) [bagDem]{Dem.};
  \node (R3) at (3.5, 0.5) [bagRep]{Rep.};

  \end{tikzpicture}
  \end{center}
\end{figure}




The {\bf{partisan preference theory}} of inflation forecast errors is similar to the partisan preference model of monetary policy described above. This theory assumes that Fed staff have a preference for more inflation averse politicians to control the executive and so produce inflation forecasts that would justify the implementation of easy monetary policy under Republican administrations and tight money under Democratic administrations, particularly as presidential elections approach. The FOMC, choosing policy based on these forecasts would then implement monetary policies to optimize its utility function, which would not need to depend upon presidential partisanship at the level of the FOMC. However, because Fed staffers prefer low inflation to high, they would not necessarily want to produce too loose/tight monetary policy over an entire four year term. Instead, they would want to encourage an economic boost (contraction) near the end of a Republican (Democratic) presidency. This implies that realized inflation would be higher than forecasted during Republican presidencies and lower than forecasted for Democratic presidencies. These effects would be particularly pronounced in the {\emph{quarters running-up to elections}} as Fed staff attempt to help their favored political party \citep{Beck1987,Grier1987}. Further, accounting for actual changes in monetary policy ought to increase the magnitude of partisan effects. This is because predictions of inflation during Republican presidencies, for example, will be lower than what the staff actually expects. If looser monetary policy is implemented in response to these low inflation forecasts than would have been chosen under the staff's true inflationary expectations, inflation will actually be higher than the staff's original forecast.

What we call the {\bf{monetary expectations theory}} is based on an assumption of partisan bias in the FOMC. It assumes that Federal Reserve economists believe members of the FOMC will engage in partisan monetary policy by lowering interest rates under conservative administrations and increasing them under liberal presidents, as \cite{Clark2012} found, and that the FOMC is doing this to manipulate election outcomes. In this formulation, the Fed staff has no preference for one party over another, but knows that the FOMC does and so formulates estimates in order to counter the FOMC's policies. If Fed economists believe that the Open Market Committee will choose systematically higher than called for interest rates during Democratic presidencies and vice versa for Republicans, then--assuming they are interested in the implementation of optimal monetary policies--they would produce forecasts that are higher than expected during Republican administrations and the lower for Democrats; the {\emph{opposite of what is expected in the partisan preference theory}}. If the FOMC fails to note the compensation made by the Fed staff, then we would expect that after accounting for implemented policies inflation forecasts would be higher than or equal to realized inflation during Republican terms and lower than or equal to forecasts under Democratic administrations. If, however, the FOMC anticipated these compensatory biases in staff forecasts, then the FOMC would discount the Greenbook estimates and continue to implement inflationary policies during Republican administrations and contradictory policies during Democratic ones. In order to combat such a strategy, it is likely tha staff would randomize their errors \note{Christopher, can you add the cite for this you mentioned on Jan 10?}, producing an uninformative signal. This would result in approximately correct inflation forecasts for both Republicans and Democrats since the forecasts would be biased in the direction of the FOMC's partisan bias rather than a ``true" no-policy-change forecast. However, we largely did not observe this and it is not illustrated above.. If the Fed staff believes that the FOMC will engage in this behavior only when presidential elections are approaching, then we would expect no partisan differences in forecasts at the beginning of a presidency but increasing divergence as the term wanes.
 
Finally, the {\bf{partisan heuristics theory}} proposes that Fed staffers incorporate a heuristic \citep[see][]{kahneman1973, tverskykahneman1974, kahneman2003} about the policy behaviors of presidents from different parties and the effects these different policies will have on the economy. Heuristics are intuitions that reduce the complexity associated with making predictions. Though they can be useful ``sometimes they lead to severe and systematic errors" \citep[][1124]{tverskykahneman1974}. In this theory, economists at the Fed have an intuition that Democrats and Republicans behave differently in government and so formulate inflationary expectations with this in mind. If this intuition does not accurately correspond to how presidents really act, or how their policies affect inflation, forecasts will be systematically biased: overestimates for Democratic presidents, underestimates for Republicans. \cite{Bartels2008} finds evidence that Democratic and Republican presidents do not, in fact, differ significantly in their overall levels of spending, so any expectation that they would pursue policies that would differentially affect inflation in the medium-run would be inaccurate.\footnote{\cite{Bartels2008} does not discuss other policies that could affect inflation in the long run, such as changes to labor and financial market regulation. These policies too would be expected to differ by party, however their lags are likely to be quite long, likely out of the forecasted time frames used in this paper.} These biases should be constant {\emph{across presidential terms}}, in contrast to the partisan preference and monetary expectation theories, which both assume an intensification of biased behavior as elections approach.

This model does not require that forecasters be conscious of the heuristics they're using, but simply for it to work its way subtly into forecasts, particularly in the subjective aspect of the ``consensus forecast" put forth in the Greenbook. Further, because this bias would not need to be conscious, the systematic error could easily go unnoticed (as mistakes could occur for any number of economic reasons that they have been trained to consider). If the bias goes unnoticed, then it will not be corrected in future inflation predictions. This differs from the {\bf{rational partisan expectations theory}} described above in the monetary policy outcome literature in two ways. First, because these beliefs are not updated to account for the lack of partisan differences in spending they are not ``rational". Second, and relatedly, this theory is based on psychological instead of game theoretic reasoning, which allows for the persistence of suboptimal strategies in a way that would be less likely in a rational choice model of this same process given the assumption that the goals of the actors are the same in the two models. % I think we might want to discuss how this contrasts with rational partisan expectations. 


%%%%%%%%%%%%%%%%%% Section 3: Research Methods %%%%%%%%%%%%%%%%%%

\section{Research Methods}

In order to test for the three potential types of partisan bias described above we use non-matched and matched data sets with parametric models \citep[see][]{Ho2007} to assess whether or not there is a partisan biases in Greenbook inflation forecasts. This section discusses the choice of methods and variables. In the following section we discuss our results.

\subsection{Matching \& Models}

We are interested in disentangling the effects of presidential party ID and elections from many background factors, such as economic shocks and FOMC policy changes, that might create inflation forecast errors. As such, we consider presidential partisan identification and elections to be `treatments' that may impact inflation forecast errors. Democratic presidencies are considered to be treatments. Republican presidencies are `controls'.\footnote{The selection of `treatment' and `control' groups in this context is arbitrary.} Quarters when a presidential election is held and the quarter before it are considered treatments and all other quarters are controls. Of course, given that we are working with observational data, other variables that have an impact on forecast errors may have {\emph{different distributions}} across the treatment and control groups \citep{Cochran1973, Diamond2012}. This makes it difficult to identify the relationships between presidential party ID, elections and errors from all of the confounding background variables.

To address this issue we follow the advice of \cite{Ho2007} to pre-process our data with matching to create data sets where the distributions of confounding variables are more even across treatment and control groups. We then run parametric analyses with the matched data sets. We used the {\tt{R}} package {\tt{MatchIt}} \citep{matchit2011} to create two matched data sets where the non-treatment covariates in the control groups closely matched those in the treatment groups. Doing this helps us isolate the effects of the `treatments' from that of the background covariates.

Formally, each unit $i$ in the data set is `assigned' to either the treatment group ($t_{i} = 1$) or the control group ($t_{i} = 0$). $y_{i}(1)$ is the potential outcome--in our case the inflation forecast error--for unit $i$ of being in the treatment group, regardless of whether or not it was observed to be in this group. $y_{i}(0)$ is the potential outcome if $i$ was not in the treatment group, regardless of its observed assignment. It is impossible to observe both $y_{i}(1)$ and $y_{i}(0)$ at the same time. Instead we observe one version of $y_{i}=t_{i}y_{i}(1)-(1-t_{i})y_i(0)$. For each $i$ there is a fixed vector of exogenous confounders $X_{i}$. Ideally $t_{i}$ and $X_{i}$ are independent. However, this is not necessarily the case. The point of matching is to reduce or eliminate the relationship between $t_{i}$  and $X_{i}$ by selecting, dropping, and/or duplicating data. Ideally this process matches one treated unit with one controlled unit that has the same values of $X_{i}$, i.e. the distribution of covariates is the same in the treated and control groups \citep{matchit2011}. This is known as ``covariate balance" \cite[1]{Diamond2012}. Using matching to balance a data set ``break[s] the link between the treatment variables and the pre-treatment controls", effectively replicating the conditions of a randomized experiment with observational data \cite[][2--3]{matchit2011}. 

Balance is usually achieved in matching through propensity scores: the probabilities that units were assigned the treatment given their covariates. The propensity score model is generally unknown \citep{Drake1993}. To find the propensity score model we use Diamond and Sekhon's \citeyearpar{Diamond2012} genetic matching method (GenMatch).\footnote{The method is implemented with {\tt{MatchIt}} The original source code for the exact matching models can be found at {\url{http://bit.ly/OFdA4u}}.} GenMatch is a multivariate method that uses an evolutionary search algorithm to automate the search for the propensity score model that creates maximum balance. This minimizes the ``challenging" process of ``manually and iteratively checking the propensity score" to determine covariate balance \citep[][2]{Diamond2012}. 


\subsection{Parametric Models}

Once we created our matched data sets we used two types of parametric models to examine the effects of our treatment variables on the continuous inflation forecast error variable.\footnote{Parametric models are from the \texttt{R} package \texttt{Zelig} \citep{Zelig2012}.} The first was normal linear regression using maximum likelihood estimation of variance (i.e. Normal Linear).\footnote{In {\tt{Zelig}} this is the {\tt{normal}} model. We also used ordinary least squares (in \texttt{Zelig}, this is the \texttt{ls} model). The results were virtually identical.} The other type was Bayesian normal linear regression\footnote{In {\tt{Zelig}} this is the {\tt{normal.bayes}} model. Bayesian normal linear regression is particularly useful for our limited sample as it makes ``valid small sample inferences via draws from the exact posterior" \citep[][38]{Zelig2012}. Please see \cite{Goodrich2007} for details about Bayesian normal linear regression.  See also \cite{Imai2008} for a discussion of how to combine non-parametric matching and parametric analysis in one research process.} Because we used  matching methods, not only do we better isolate the treatments' effects from those of the background variables, but we also reduce our estimated causal effects' dependence on the type of model we choose \cite[200--201]{Ho2007}.

The generic parametric model we used is given by
%
\begin{equation}
    E_{q} = \alpha + \beta T_{q} + \beta X_{q} + \epsilon,
\end{equation}
%
where $T_{q}$ is the treatment for quarter $q$ and $X$ is a vector of covariates. 

\subsection{Variables}

In Section \ref{ForecastAcc} we discussed our dependent variable--inflation forecast errors. To examine possible partisan biases we are interested in whether US presidents' partisan identity and/or the existence of an upcoming presidential election affect these errors. To do this we created {\bf{president party identification}} and {\bf{election period}} variables. The president party ID variable is 1 when the president is a Democrat and 0 when he is a Republican. Since forecast error data is released on a quarterly basis, we consider a president to be sitting from the first quarter after the election.\footnote{Elections are held almost at the midpoint--early November--of an election year's fourth quarter. Presidents are sworn into office near the beginning--20 January--of the following year's first quarter.} We consider quarters to be in the election period either if the presidential election is held in that quarter or in the previous three quarters.\footnote{If $q_{e}$ is a quarter with an election then we code quarters $q_{e}$, $q_{e-1}$, $q_{e-2}$, and $q_{e-3}$  election quarters.} The economic voting literature indicates that it is economic performance in the 6-12 months preceeding an American presidential election that seem to matter most for the election's outcome \citep[c.f]{Gelman1993}.

To further examine whether or not Federal Reserve staff were taking into consideration an electoral business cycle either due to a partisan preference or the nearness of an election, we include a variable of the {\bf{quarters until the presidential election}}. This simply counts down from the quarter after the previous election.\footnote{There are 15 quarters before a United States presidential election quarter.} The quarters that included presidential elections are coded as 0. 

The partisan preference and monetary expectations theories both posit that president's party ID and elections have a non-linear interactive relationship with forecast errors. To examine this possibility we include an interaction between the president party ID variable and the square of election period variable in the analyses.\note{Huh? Isn't the election period variable simply a 1 or 0? Why is it squared?}

The United States president does not set the level of government expenditure--one of the main non-monetary policy causes of inflation--by themselves. Instead, the president is jointly constrained by the two houses of Congress. To examine whether or not Federal Reserve staff are taking into consideration the partisan composition of Congress as well as the president's party identification, we include a variable of {\bf{Democratic legislators as a proportion of Republican legislators}} in the House of Representatives as well as a similar variable for the partisan composition of the Senate.\footnote{Data on the number of legislators with Republican and Democratic party IDs was found at infoplease. See {\url{http://www.infoplease.com/ipa/A0774721.html}}. Accessed May 2012.} 

Because each chamber of Congress acts as a veto player on the main fiscal expenditures, a Congressional effect on errors likely works through an interaction between the partisan ID of Congress and the presidency. There are two types of interaction effects provided in the literature. The first interaction possibility is that Federal Reserve staff with simple rational partisan expectations would presume that a Democratic president would be able to get policies closer to their ideal point when there is a Congress with similar preferences. If a Democratic president had chambers of Congress controlled by Democrats, presumably Federal Reserve staff would expect even higher fiscal expenditures and therefore even higher inflation. Conversely, Republican presidents with a Republican-controlled Congress may be even better at cutting spending, leading to even lower inflation.

The second interaction possibility is based largely on Krause's \citeyearpar{Krause2000} work about the affect of partisan divisions on fiscal deficits in the United States. He finds partisan fragmentation can play a role in increasing Federal deficits. Higher political conflict, he argues, ``results in equilibrium fiscal outcomes that favor greater spending and/or a willingness to lower taxes since politicians will exhibit a greater proclivity in providing voters with program benefits and to delay its payment" \citep[][542]{Krause2000}. Because of this Federal Reserve staff may anticipate higher government borrowing when the presidency and houses of Congress are controlled by different parties. We are therefore agnostic about the theoretical direction of this interaction.

If prediction errors are largely the result of systematically biased economic forecasting models we would expect errors to change overtime as the models changed. In particular, we would expect a decrease in the magnitude of errors to occur around 1996 when the Federal Reserve Board's new US and Global Behavioral Equation Model was introduced. To examine this we include a {\bf{FRB/Global Model}} dummy variable. It equals one for all quarters from the first quarter of 1996 onward. It is zero otherwise.

Greenbook forecasts are based on the assumption that monetary policy will not change between when the prediction is made and the time period it is predicting.\footnote{While the Fed staff also produce forecasts under alternative monetary policies in the so-called ``Bluebook," these data are not available in a readily usable format (i.e., not in a dataset but only in the orginal reports themselves) and thus are not used in the forecasting error literature.} Forecast errors may occur if monetary policy changes in the interim. If this is the case monetary policy changes would have a negative relationship with forecast errors. When the FOMC raises interest rates inflation may decline, causing the original forecast to have been too high and vice versa. To control for monetary policy changes we include a variable of {\bf{standardized changes to the discount rate}} from the quarter the Greenbook prediction was made to the quarter it is predicting.\footnote{We averaged the discount rate over each quarter. Then we used the average discount rate $D$ in each quarter $q$ to create the variable $\Delta D_{q}$ using the simple formula: $\Delta D_{q} = \frac{D_{q} - D_{q-2}}{D_{q}}$. Note that the Federal Reserve changed how it used the discount rate and referred it at the beginning of 2003. To address this issue we primarily used data on the United States' discount rate recorded by the International Monetary Fund. Their data only goes back to the fourth quarter of 1982. So, before that we use the Federal Reserve's measure of the discount rate. Both of these variables are found in the FRED database at the St. Louis Federal Reserve (accessed July 2012). } The discount rate is one of the Federal Reserve's main tools for influencing the interest rate, especially the Fed funds rate.\footnote{A similar {\bf{relative changes in the Fed funds rate}} variable was included in some preliminary analyses. However, it did not change the results substantially and was estimated to have a similar effect on errors as the discount rate variable.}

To examine if Federal Reserve inflation forecast errors are affected by levels of government expenditure we include the percentage of {\bf{current government expenditure to GDP}}, {\bf{government debt to GDP}},\footnote{Results for debt to GDP are not shown because it was never statistically significant in any of the models} and \textbf{deficit to GDP}. Expenditure and debt are on a quarterly basis, while federal deficits are measured annually.\footnote{All three of these variables are from the FRED database, accessed October 2012 and January 2013.} 

To examine how broader economic factors may be related to forecast errors we include variables of the {\bf{GDP output gap}}, {\bf{unemployment rate}} and {\bf{recession}}. The GDP output gap is the potential GDP as a percentage of real GDP. It is in nominal terms. The recession variable is a dummy for whether or not the United States was in a recession.\footnote{All three of these variables are from the FRED database.Accessed June and October 2012.} 

Finally, we include a variable for the sitting {\bf{Federal Reserve Board Chair}}.\footnote{Chairs for the years in our analysis are William McChesney Martin, Jr., Arthur Burns, G. William Miller, Paul Volcker, Alan Greenspan, and Ben Bernanke.}

\section{Results}

In this section we graphically present results from a number of parametric model specifications and discuss our findings. Coefficient estimate tables can be found in the Appendix (see tables \ref{OutputNL}, \ref{OutputEL}, \ref{OutputPL}, \ref{OutputNB}, and \ref{OutputPB}). In general there is very little difference between the coefficients estimated using normal linear and Bayesian linear regression models (see Figure \ref{CoefComparePlots}). Usually a few more variables--such as the output gap and discount rate change--were `statistically significant' at the standard 0.05 level in models using unmatched data compared to estimates from the matched data sets. 

We primarily diagnose the presidential party ID matching models with a propensity score distribution--the probability of a quarter being in the `treated' group given its covariates--as well as quantile-quantile plots to diagnose whether or not each covariate in the matched data sets is balanced \citep{Ho2007}.\footnote{Please see Figure \ref{PresPropensityScores} for the propensity score distributions in our matched data sets. The quantile-quantile plots are not shown, but can easily be created by running the original matching models in our main analysis source code file. The file is available at: \url{http://bit.ly/OFdA4u}.}  We are unable to achieve covariate balance for the Congressional interaction terms\footnote{The presidential party ID and election period interaction does balance.} and the Federal Reserve Chair variable. Chairs in our data set in the pre-Volcker/Greenspan era as well as current Chair Ben Bernanke were in office for very few observed quarters, making it difficult to match them. It is unlikely that  covariate balance could be achieved on these interactions given our relatively small sample size. Because of this, we only discuss Congressional interaction terms and Chair variable estimates from the non-matched data sets.}

We remove quarters from the sample where forecasters would not have known who the president would be because the president had not yet been elected for that quarter. For models where the dependent variable is forecasts made two quarters beforehand this means removing the 2 quarters following each presidential election.\footnote{In this case 19 quarters are removed.} Results from these restricted data sets are fairly similar to those from the full data set.

Almost all other parametric model specifications\footnote{\ref{OutputPL} does show interaction estimates from matched data sets for comparison with the non-matched results in Table \ref{OutputNL}.} are run with both the matched and unmatched data sets.\footnote{In all of the Bayesian regressions we use the {\tt{Zelig}} default 1,000 MCMC burn-in iterations and 10,000 iterations after burn-in. We use the Heidelberger-Welch diagnostic to examine whether or not the Markov Chains converged to their stationary distributions.}





























