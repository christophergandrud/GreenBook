%%%%%%%%%%%%%%%%%%%%%%%%%%%
% Inflated Expectations
% Cassandra Grafström and Christopher Gandrud
% 3 November 2013
%%%%%%%%%%%%%%%%%%%%%%%%%%%%

% !Rnw weave = knitr

\documentclass[a4paper]{article}
\usepackage{fullpage}
\usepackage[authoryear]{natbib}
\usepackage{setspace}
    \doublespacing
%\usepackage[usenames,dvipsnames]{xcolor}
\usepackage{hyperref}
\hypersetup{
    colorlinks,
    citecolor=black,
    filecolor=black,
    linkcolor=cyan,
    urlcolor=cyan
}
\usepackage{dcolumn}
\usepackage{booktabs}
\usepackage{url}
\usepackage{tikz}
\usepackage{todonotes}
\usepackage[utf8]{inputenc} 
\usepackage{graphicx}

% Set knitr global options



%%%%%%% Title Page %%%%%%%%%%%%%%%%%%%%%%%%%%%%%%%%%%%%%%%%%%%%
\title{Inflated Expectations: How government partisanship shapes monetary policy bureaucrats' inflation forecasts}

\author{Christopher Gandrud \\
                {\emph{Hertie School of Governance}}\footnote{Postdoctoral Researcher. Friedrichstra{\ss}er 180. 10117 Berlin, Germany. Email: \href{mailto:christopher.gandrud@gmail.com}{gandrud@hertie-school.org}. Thank you to the Mark Hallerberg and the Fiscal Governance Centre at the Hertie School of Governance for comments and support. Thank you also to Leonardo Baccini, Vincent Arel-Bundock, Cheryl Schonhardt-Bailey, Tom Stark, and seminar participants at the Hertie School of Governance, London School of Economics, and Yonsei University. This paper was written using {\tt{knitr}} \citep{knitr2013}. It can be entirely replicated from data, analysis source code, and markup files available at: {\url{https://github.com/christophergandrud/GreenBook}}.} \\
                and \\
            Cassandra Grafstr\"{o}m \\
                {\emph{University of Michigan}}\footnote{Ph.D Candidate. 5700 Haven Hall, 505 S. State Street
                Ann Arbor, MI 48109-1045. USA. Email: \href{mailto:cgrafstr@umich.edu}{cgrafstr@umich.edu}.}}

\begin{document}

\maketitle

%%%%%%% Abstract %%%%%%%%%%%%%%%%%%%%%%%%%%%%%%%%%%%%%%%%%%%%
\begin{abstract}

Governments' party identifications can indicate the types of economic policies they are likely to pursue. A common rule of thumb is that left-party governments are expected to pursue policies for lower unemployment, but which may cause inflation. Right-party governments are expected to pursue lower inflation policies. How do these expectations shape the inflation forecasts of monetary policy bureaucrats? If there is a mismatch between the policies bureaucrats \emph{expect} governments to implement and those that they \emph{actually} do, forecasts will be systematically biased. Using US Federal Reserve Staff’s forecasts we test for executive partisan biases. We find that irrespective of actual policy and economic conditions forecasters systematically overestimate future inflation during left-party presidencies and underestimate future inflation during right-party ones. Our findings suggest that partisan heuristics play an important part in monetary policy bureaucrats' inflation expectations.

\end{abstract}

\begin{description}
  \item [{\textbf{Keywords:}}] forecast bias, Federal Reserve bureaucrats, rational partisan cycle, heuristics, inflation, monetary policy
\end{description}

\vspace{0.3cm}


Monetary policy is an inherently forward looking enterprise. Beliefs about the economy's future course guide the setting of interest rates. Government policies significantly affect changes in growth, inflation, and unemployment. Further, a government's party identification can serve as a cue for the types of economic policies that it is likely to pursue during its tenure. A common rule of thumb or ``heuristic'' for the United States is that right-leaning Republicans are expected to pursue policies associated with lower inflation and left-leaning Democrats are expected to pursue policies associated with lower unemployment, but more inflation \cite[see][]{Samuelson1977,HibbsJr1977}. Recent evidence suggests, however, that the real differences in economic policies implemented by the two parties are quite minimal \citep{Bartels2008}. Pursuing monetary policy based on expectations of differences in partisan behavior, as opposed to their general similarities in reality, could lead to systematic mistakes in the setting of monetary policy. It is therefore important to ask how US Federal Reserve Staff incorporate the government's partisan composition when forming expectations of future inflation.

We provide strong evidence that Fed internal inflation forecasts--the forecasts on which monetary policy decisions are based--are heavily influenced by an inaccurate presidential partisan heuristic. These forecasts consistently predict that inflation will be lower than it turns out to be under Republican presidencies and that inflation will be higher than it turns out to be under Democratic administrations. Even accounting for changes in monetary policy and a variety of other economic and political factors, Federal Reserve economists over-shoot inflation forecasts for Democrats and under-shoot for Republicans. We find that previous literature on how partisanship may affect monetary policy--primarily work done on partisan preferences \citep{Clark2012,Hakes1988,Sieg1997,Tootell1996} and rational monetary policy expectations \citep{Alesina1987,Alesina1991,Hibbs1994}--is not as useful as a partisan heuristic approach for explaining these predictive failures.

In this paper we first briefly describe bureaucratic inflation forecasting at the US Federal Reserve, why it is important for monetary policy-making, and previous research on sources of bias in it. Academic scholarship on partisanship and Fed forecasting has largely been non-existent. So we introduce the presidential partisan heuristic approach to explaining prediction errors. We also derive major alternative hypotheses about how partisan control of the presidency might shape inflation forecasts by Federal Reserve Staff from the literature on partisanship and monetary policy-making. We then discuss how to measure inflation forecast errors. Using Fed Staff's ``Greenbook'' forecasts we demonstrate that there does appear to be a presidential partisan bias. To understand why these errors exist we test the theories of partisan bias with a series of regression models using both unmatched and matched data on inflation forecast errors from the 1970s through 2007.\footnote{This is the most complete data set currently available to the public.} Our findings suggest that even when controlling for a number of important economic and political factors, Greenbook forecasts show a distinct presidential partisan bias across presidential terms, not just in the run up to elections, as competing theories would suggest. Rather than being caused by electoral preferences or partisan monetary policy expectations--we find strong evidence supporting our hypothesis that the bias is caused by an incorrect partisan heuristic Fed Staff hold about administrations' likely effect on inflation. Our finding highlights psychological aspects of how bureaucrats deal with uncertainty that, though they have been researched extensively in the behavioral economics literature, have previously been ignored by researchers empirically examining monetary policy bureaucracies. In the conclusion we discuss the implications of these findings for monetary policy, election outcomes, and future research directions. 

%%%%%%%%%%%%%% Section 1: Forecasting Inflation at the Fed %%%%%%%%%%%%%%%%%%%%%

\section{Bureaucratic Inflation Forecasting in the United States}

Inflation forecasting is crucial for enabling monetary policy makers to maintain price stability. The primary instrument of monetary policy--interest rates--``only exert their full full effect on \ldots inflation with some considerable delay'' \cite[59]{Goodhart2001}. The US Federal Reserve's Greenbook forecasts are an component of bureaucratic inflation forecasting in the United States. Prior to every Federal Open Markets Committee (FOMC) meeting Federal Reserve Staff create a document called the ``Current Economic and Financial Conditions''--Greenbook--that contains information on recent behavior and forecasts of various macroeconomic aggregates assuming no monetary policy change.\footnote{Greenbook data can be found at {\url{http://www.phil.frb.org/research-and-data/real-time-center/greenbook-data/philadelphia-data-set.cfm}} (accessed March 2013). Greenbook forecasts are currently available to the public for each quarter from the fourth quarter of 1965 through the end of 2007. There is a five year lagged release schedule. Also, some forecasts are not available for the entire period.} Federal Reserve Staff make forecasts of various elements of the US and global economies so that the FOMC can make policies appropriate to fulfill the Fed's dual mandate of maintaining maximum employment and price stability.

As \cite{Svensson2005} notes, the accuracy of forecasts is essential to the effectiveness of monetary policy. The FOMC directly uses these forecasts to determine the appropriate monetary policy to pursue. Greenbook forecasts are given to FOMC members one week before each meeting. Staff also present the Greenbook forecasts during FOMC meetings. Expectations, directly influenced by Greenbook forecasts, play a very important part in FOMC decision-making. From FOMC minutes we know that members spend a considerable amount of time discussing prospective economic conditions. In fact much of the FOMC meeting time is used to discuss what economic conditions are likely to be rather than the relative desirability of various outcomes. \cite[230]{RomerRomer2008}. Greenbook forecasts directly frame these discussions. What members believe will happen in the future directly influences their policy choices. Higher inflationary expectations increase the likelihood of a member supporting tightening policy in order to slow inflation and an overheating economy; low inflationary expectations increase the likelihood of a preference for loosening monetary policy to bolster growth and employment, all else equal. Therefore, to act as useful baseline for FOMC decision-making it is important that Greenbook forecasts accurately predict future inflation.

The study of Fed inflation forecasts and their accuracy has been almost exclusively contained within economics with the main concern being the rationality of forecasts \cite[e.g.][]{Capistran2006, Romer2000} and the performance of the Fed's forecasts relative to market forecasts \cite[e.g.][]{Faust2007, Gamber2009}. While some studies in the economics literature have examined the biases of particular time periods \cite[e.g.][]{Capistran2006} or bank presidents \cite[e.g.][]{Havrilesky1995}, in our search none considered how government partisanship affects expectations about future inflation at the Federal Reserve.
 
It is important to note that finalized Greenbook forecasts are ``consensus'' forecasts combining \emph{both} econometric models and the professional opinions of forecasters about likely changes in the economy's trajectory not necessarily picked up in these models \citep{Karamouzis1989,Reifschneider1997}. Preferences and/or beliefs about government partisanship, rather than just econometric models based on explicit assumptions, could therefore directly shape Greenbook forecasts. 

The idea that politicians of different partisan stripes might behave differently in office and that their behavior might have different effects on future economic output and inflation is largely uncontroversial for political scientists.\footnote{See \cite{Bartels2008} for evidence on the similarities and differences of Democrats and Republicans in office.} However, political scientists have largely not explored whether or how these differences affect Federal Reserve inflation forecasts. The closest attempt made to explore partisan biases in Fed forecasts that we know of was done by \cite{Frendreis2000}. Using simple frequency tables and yearly data, they examined the accuracy of forecasts by the Congressional Budget Office (CBO), presidential administrations', and Federal Reserve Staff. Though they listed the accuracy of Fed inflation forecasts as measured by absolute mean error for the whole period 1979-1997,\footnote{They found that absolute mean errors were similar to the CBO's and less than administrations'.} they \emph{did not examine partisan biases} or any other cause of inaccuracies in Fed forecasts. Their study of partisan biases was entirely confined to a comparison of CBO and administrations' forecasts. 

%%%%%%%%%%%%%%%%%%%%%%%%%%%%% Section 2: Possible Explanations of Fed Inflation Forecast (In)accuracy %%%%%%%%%
\section{Possible Explanations of Fed Inflation Forecast Inaccuracy}

There is an extensive literature looking for partisan effects in monetary policy decisions, but few studies examining the impact of presidential partisanship on central banks' predictions of inflation. In this part we first introduce our presidential partisan heuristic approach to understanding an important component of Fed Staff's inflation forecasting errors. A theory is useful to the extent that it explains a phenomenon better than the most plausible alternatives. So, after introducing partisan heuristics, we posit alternative explanations for the errors that build on previous approaches to understanding the relationship between partisanship and monetary policy decisions that we reformulate for the issue of bureaucratic inflation forecasting.

\subsection{Partisan Heuristics}

Previous research has shown that (a) central bankers use heuristics or `rules of thumb'--such as Okun's law and the Taylor Rule--to help them understand complex and uncertain economic conditions and (b) that people's economic expectations are shaped by the government's partisan orientation. To understand how presidential partisanship may impact inflation forecasting errors we propose combining these two insights into one coherent approach that we call {\bf{presidential partisan heuristics}}. 

\paragraph{Heuristics \& inflation expectations}

Let's start by reviewing the research on how Federal Reserve bureaucrats use heuristics generally to help them predict future inflation. Let's assume that Fed staff have an interest and/or incentive (e.g. because of professional socialization and performance reviews) to make the most accurate forecasts possible. Given that numerous studies have found that Fed Staff forecasts are more accurate than private sector forecasts \cite[see][]{Romer2000,Gamber2009}, this is a reasonable assumption.  

However, even if we assume that Fed Staff want to make very accurate predictions and they have access to high quality information, there is still considerable uncertainty about future inflation due to the complexity of economic relationships that create it and the difficulty of adequately observing these relationships. In fact \cite{Gamber2009} find that though the overall level of inflation in the United States has decreased since the early 1980s inflation has actually become more difficult to predict. They argue that decreased inflation variability is the result of a decline in the variability of the predictable component of inflation. [[[Explain]]] Even if Fed Staff with relatively high quality information want to accurately predict inflation, it is difficult for them to do so. This is precisely the type of situation where we would expect actors to supplement their forecasting models with heuristic judgments in an attempt to create better expectations. Heuristics are intuitions that reduce the complexity associated with making predictions. Considerable psychological and economic research has shown that across a wide range of settings, including with area experts, that people reduce the complexity of predicting uncertain values by incorporating simpler heuristics \citep[see][]{kahneman1973,tverskykahneman1974,Tversky1983,Kahneman2002,kahneman2003}. 

The role of judgment and heuristics to help forecast inflation in light of its complexity has been widely discussed and researched among Federal Reserve staff members themselves. For example, \cite{McNees1990} used individual forecasts to examine situations under which forecasters' judgmental adjustments improved (or didn't) economic forecasts. He found that though judgment can add important information to forecasts, forecasters have a tendency to over adjust their models based on their own judgments. \cite{Orphanides2008} found that FOMC members used a Taylor `rule of thumb' \citeyearpar{Taylor1993} with expected economic conditions when making interest rate decisions. Work by \cite{Tillmann2010Philips} and \cite{KnotekII2007} has found that FOMC members use Phillip's Curve and Okun's Law rules of thumb to predict the relationship between unemployment, inflation, and growth. They find that the specifications of these rules have been updated over time as the relationships between these quantities have changed.

Overall, the focus in this work has been on rules of thumb based on economic factors. Because monetary policy bureaucrats often explicitly examine the relationships between these quantities--even if they don't fully understand them and therefore need to rely on rule of thumb based judgments--they are likely to update their rules as new information becomes available. 

\paragraph{Partisan inflation expectations}

However, inflation is not a purely economic phenomenon in the sense that government policies beyond monetary policy narrowly defined as setting interest rates can affect future inflation. Partisan theories predict that these policies will vary systematically by governments' partisan affinity \citep{Samuelson1977,HibbsJr1977}. In sum, left-leaning Democratic presidents are believed to pursue policies that increase inflation and right-leaning politicians are expected to pursue policies that reduce inflation. [[[CASSIE CAN YOU FILL IN WITH A BREIF STORY?]]] So it would be reasonable to include information about what policies a government is expected to pursue when forecasting inflation.

To our knowledge the possibility of a \emph{partisan} heuristic among monetary bureaucrats has not been examined. However, previous research has found that members of the public do use presidential partisan heuristics to predict future economic conditions, including inflation. Examining survey data of general assessments of economic health, \cite{Duch2000} found that, rather than being objective, people perceived both current and future economic conditions differently based on personal characteristics, including whether or not they shared a partisan affinity with the government. \cite{Snowberg2007} used faulty data from presidential election prediction markets before the 2004 election to examine how partisan expectations move economic indicators. They found that a number of indicators, such as bond yields and exchange rates, moved in directions consistent with standard partisan expectations of policy changes under Democratic and Republican presidents when they thought that candidates from these parties would win.\footnote{Unfortunately, they did not have an economic indicator that could capture changes in inflation expectations.} \cite{Fowler2006} looked specifically at how prediction markets affected futures markets for nominal interest rates. When the probability of a Democrat winning the US Presidency (or Congress) increased, so did interest rate futures. He interprets this finding to suggest that actors expected inflation to increase under a Democratic government. This confirmed a finding using similar methods by \cite{Alesina1997}. \cite{Berlemann2006} extend evidence for a partisan expectations effect to five other developed countries in addition to the United States.

\paragraph{Presidential partisan heuristics \& inflation forecasting errors}

Drawing on these finding we argue that Fed Staff forecasters are likely to incorporate heuristic about the likely affect of public policies on inflation from president's party identification.

Why focus on presidents' party identification? Incorporating information about non-monetary policy actions is particularly difficult in the American context. There are many layers of government--national, state, cities--that make policies which can impact inflation. Even within the national government there are multiple branches often controlled by different parties that can influence policies. In this case, forecasters may intuitively use a heuristic based on the partisan identification of arguably the single most influential actor in this system--the president as a way to simplify the complex relationships between politics and inflation.

Presidential partisanship could be used as a ``prototype heuristic'' to help forecasters predict inflation. Prototype heuristics are a general class of heuristics where people substitute the mean or exemplar attribute--prototype--of a category for what they are trying to determine \cite[1463]{kahneman2003}. Prototypes have been found to impact judgments on problems as diverse as the pricing of goods, the effectiveness of painful medical procedures, and the prediction of floods by professional forecasters. Presidents' partisan identifications are easily observable. The impact of public policy on inflation is not. Fed Staff could use prototypical information about presidential party ID, which is intuitive and readily available, as a substitute for more complete, but more difficult to obtain information about what public policy is likely to be and how it will impact inflation. 

The `prototypical' Democratic president has been believed to be less concerned with low inflation than the prototypical Republican president, who would therefore pursue policies that dampen inflation. Therefore forecasters using a presidential partisan prototype heuristic would predict inflation to be higher under a Democratic president and lower under a Republican, all else equal. 

Though they can be useful, ``sometimes they lead to severe and systematic errors'' \citep[][1124]{tverskykahneman1974}. In this theory, economists at the Fed have an intuition that Democrats and Republicans behave differently in government and so formulate inflation expectations with this in mind. If this intuition does not accurately correspond to how presidents act, or how their policies impact inflation, forecasts will be systematically biased: overestimates for Democratic presidents, underestimates for Republicans. \cite{Bartels2008} finds evidence that Democratic and Republican presidents do not, in fact, differ significantly in their overall levels of spending, so any expectation that they would pursue policies that would differentially affect inflation in the medium-run would likely be inaccurate.\footnote{\cite{Bartels2008} does not discuss other policies that could affect inflation in the long run, such as changes to labor and financial market regulation. These policies too would be expected to differ by party, however their lags are likely to be quite long and out of the forecasted time frames used in this paper.} These biases should be constant {\emph{throughout a president's term}}. As we will see, this prediction contrasts with the partisan preference and monetary expectation theories, which both assume an intensification of biased behavior as elections approach. Figure \ref{ExpectGraphs} shows an illustrated comparison of the three theories we set out.

Our model does not require that forecasters be conscious of the heuristic they're using. It can simply work its way subtly into forecasts, particularly in the subjective aspect of the ``consensus forecast" component of the Greenbook. If the models do not conform with other expectations about the economy's current course, based in part on these subtle partisan heuristics, the consensus forecast will be modified accordingly. Further, because this bias would not need to be conscious, the systematic error could easily go unnoticed (as mistakes could occur for any number of idiosyncratic or economic reasons). If the bias goes unnoticed, then it will not be corrected in future inflation predictions.\footnote{This assumption is in contrast to \cite{Grauwe2011}, who assumes that actors actively observe their heuristics and adapt them through trial and error. He does not provide empirical evidence supporting this assumption, however.} This differs from the rational partisan expectations theory \cite[e.g.][]{Alesina1987,Alesina1991,Alesina1997,Hibbs1994}, that we discuss in more detail below. First, because these beliefs are not updated to account for the lack of partisan differences in spending they are not ``rational". Second, and relatedly, this theory is based on psychological instead of game theoretic reasoning, which allows for the persistence of suboptimal strategies in a way that would be less likely in a rational choice model of this same process given the assumption that the goals of the actors are the same in the two models.





% Define colors for figure
%% See: http://colorbrewer2.org/
\definecolor{DEM}{HTML}{2259B3}
\definecolor{REP}{HTML}{C42B00}

\begin{figure}
  \caption{Stylized Partisan Inflation Forecast Error Predictions}
  \label{ExpectGraphs}
  \begin{center}

    \vspace{0.25cm}

    \tikzstyle{bagMain} = [text width = 5cm]
    \tikzstyle{bagDem} = [text = DEM]
    \tikzstyle{bagRep} = [text = REP]
    
    \tikzstyle{DemLine} = [draw, 
                          color=DEM,
                          opacity=0.9,
                          line width=1.5mm]

    \tikzstyle{RepLine} = [draw, 
                          color=REP,
                          opacity=0.9,
                          line width=1.5mm]

\begin{tikzpicture}

  %%%% Partisan Heuristics
  \node (PP) at (-7, 5) [bagMain]{{\bf{Partisan Heuristics}}};
  \node (E) at (-10.5, 3.25) [bagMain, rotate=90]{{\emph{Forecast Error}}};
  \node (T) at (-7.3, -0.5) [bagMain]{{\emph{Duration of Pres. Term}}};
  
  \draw (-10, 0) -- (-6, 0);
  \draw (-10, 0) -- (-10, 4);


  \draw[DemLine] (-9.5, 3) -- (-6.5, 3); 
  \draw[RepLine] (-9.5, 1) -- (-6.5, 1); 

  \node (R1) at (-6.5, 0.5) [bagRep]{Rep.};
  \node (D1) at (-6.5, 3.5) [bagDem]{Dem.};
  
  %%%% Partisan Preferences
  \node (PP) at (-2, 5) [bagMain]{{\bf{Partisan Preferences}}};
  
  \draw (-5, 0) -- (-1, 0);
  \draw (-5, 0) -- (-5, 4);
  
  \draw[RepLine] (-4.5, 1.9) -- (-2.5, 1.9); 
  \draw[DemLine] (-4.5, 2.1) -- (-2.5, 2.1); 

  \draw[RepLine] (-2.52, 1.9) -- (-1.5, 1); 
  \draw[DemLine] (-2.52, 2.1) -- (-1.5, 3); 


  \node (D2) at (-1.5, 0.5) [bagRep]{Rep.};
  \node (R2) at (-1.5, 3.5) [bagDem]{Dem.};
  
  
  %%%% Monetary Expectations
  
  \node (PP) at (3, 5) [bagMain]{{\bf{Monetary Expectations}}};
  
  \draw (0, 0) -- (4, 0);
  \draw (0, 0) -- (0, 4);

  \draw[RepLine] (0.5, 2.1) -- (2.53, 2.1); 
  \draw[DemLine] (0.5, 1.9) -- (2.53, 1.9); 

  \draw[DemLine] (2.53, 1.9) -- (3.23, 1); 
  \draw[RepLine] (2.53, 2.1) -- (3.23, 3); 

  \node (D3) at (3.5, 3.5) [bagRep]{Rep.};
  \node (R3) at (3.5, 0.5) [bagDem]{Dem.};
  


  \end{tikzpicture}
  \end{center}
\end{figure}




%%%%%%%%%%%%%%%

The usefulness of a theory is partially demonstrated by how well it explains events relative to its major competitors. Though no previous studies have examined partisan biases in inflation forecasts, competing theories can be derived from studies that have looked for evidence of partisan preferences manifesting themselves in the FOMC's monetary policy outcomes. Two key strains in this literature have looked for partisan effects as either resulting from a preference for one party over another by members of the FOMC or an expectation that once in office the parties will engage in systematically different policies that will influence inflation, leading the FOMC to support more preferred policies and attempt to inhibit less preferred ones. 

The preference arguments about monetary policy-making assume that a conservative central banker will prefer the election of politicians who hold more similar inflationary preferences (i.e., those with a stronger preference for low inflation) and enact policies to bolster their preferred candidate's prospects of being elected. In the US this would mean that the FOMC would implement policies that supported the electoral prospects of Republican incumbents and harm the electoral prospects of Democratic incumbents \citep{Clark2012,Hakes1988,Sieg1997,Tootell1996}.

Building on this approach, a {\bf{partisan preference theory}} of inflation forecast errors assumes that Fed Staff have a preference for more inflation averse politicians to control the executive and so produce inflation forecasts that would justify the implementation of easy monetary policy under Republican administrations and tight money under Democratic administrations, particularly as presidential elections approach. The FOMC, choosing policy based on these forecasts would then implement monetary policies to optimize its utility function, which would not need to depend upon presidential partisanship at the level of the FOMC. However, because Fed Staffers prefer low inflation to high, they would not necessarily want to produce too loose/tight monetary policy over an entire four year term. Instead, they would want to encourage an economic boost (contraction) near the end of a Republican (Democratic) presidency. This implies that realized inflation would be higher than forecasted during Republican presidencies and lower than forecasted for Democratic presidencies. These effects would be particularly pronounced in the {\emph{quarters running-up to elections}} as Fed Staff attempt to help their favored political party \citep{Beck1987,Grier1987}. Further, accounting for actual changes in monetary policy ought to increase the magnitude of partisan effects. This is because predictions of inflation during Republican presidencies, for example, will be lower than what the staff actually expects. If looser monetary policy is implemented in response to these low inflation forecasts than would have been chosen under the staff's true inflationary expectations, inflation will actually be higher than the staff's true beliefs about inflation under no change in monetary policy.

The rational partisan expectations literature on monetary policy-making assumes that central bankers do not have an innate preference for one party over another, but instead expect Democrats and Republicans to behave differently in office \citep{Alesina1991,Hibbs1994}. It is these behavioral expectations that would lead to different monetary policies under Democratic and Republican presidencies, with the former expected to engage in more expansionary and inflationary policies than the latter. In order to stave off higher inflation under a Democrat the Fed would tighten monetary policy; because Republicans are expected to prefer lower inflation, they will pursue policies in support of that goal and so the FOMC can accommodate Republican presidents' policies without fear of stoking inflation. This argument is again based on the assumed preferences of partisans, but does not require the FOMC to be politically biased as the former does. 

What we call the {\bf{monetary expectations theory}} is based on an assumption of partisan bias in the FOMC rather than among the staff. It assumes that Federal Reserve economists believe members of the FOMC will engage in partisan monetary policy by lowering interest rates under right-leaning administrations and increasing them under left-leaning presidents, as \cite{Clark2012} found, and assumes that the FOMC is doing this to manipulate election outcomes. In this formulation, the Fed Staff has no preference for one party over another, but knows that the FOMC does and so formulates estimates in order to counter the FOMC's policies. If Fed economists believe that the FOMC will choose systematically higher-than-called-for interest rates during Democratic presidencies and vice versa for Republicans, then--assuming they are interested in the implementation of optimal monetary policies--they would produce forecasts that are higher than expected during Republican administrations and the lower for Democrats; the {\emph{opposite of what is expected in the partisan preference theory}}. If the FOMC fails to note the compensation made by the Fed Staff, then we would expect that after accounting for implemented policies inflation forecasts would be higher than or equal to realized inflation during Republican terms and lower than or equal to forecasts under Democratic administrations.\footnote{This is illustrated in the center panel of Figure \ref{ExpectGraphs}.} If, however, the FOMC anticipated these compensatory biases in staff forecasts, then the FOMC would discount the Greenbook estimates and continue to implement inflationary policies during Republican administrations and contractionary policies during Democratic ones. If the staff likewise know that they are not being listened to they may randomize their errors, producing an uninformative signal \citep{Crawford1982}. This would result in approximately similar inflation forecast errors for both Republicans and Democrats. However, we largely did not observe this (see below). If the Fed Staff believes that the FOMC will engage in partisan pumping only when presidential elections are approaching, then we would expect no partisan differences in forecasts at the beginning of a presidency but increasing divergence as the term wanes.

\subsection{Econometric Models \& Accuracy}

Before empirically digging into partisan explanations of forecast errors, which would largely be the result of Federal Reserve Staff judgement, it is worth examining the possibility that forecast inaccuracy is the result of systematic errors in the staff's predictive econometric models. Federal Reserve Staff have primarily used two sets of econometric models during the period for which Greenbook data is available.\footnote{This subsection draws heavily on Brayton et al.'s \citeyear{Brayton1997} detailed description of the changes to Federal Reserve forecasting models that took place in 1996.} 

The first simultaneous equation model of the US and world economies were developed and adopted by the Federal Reserve between 1966 and 1975. This model was based on adaptive expectations and largely extrapolated future behavior of the economy from its recent past behavior. New models of the American and world economies' near-term trajectories were introduced in the 1990s, fully replacing the older model in 1996. The Federal Reserve Board US model (FRB/US) and its counterpart for the global economy (FRB/Global) explicitly consider the role of economic expectations in economic behavior. The foundational assumption of adaptive expectations in the old model was replaced with rational or model-consistent expectations. In these models prices are sticky and aggregate demand determines short-run output. Furthermore, monetary policy's effects on the economy are extensively modeled. 

Presumably, the move to rational expectations would improve forecast accuracy relative to the earlier period. The goal of incorporating forward looking actors into the models was to account for an important source of endogeneity in earlier models that could lead to overestimates of important economic indicators under some circumstances and underestimates of those same indicators under others. None of these over or underestimates, however, ought to have been linked to the party of the president. We would, however, expect that the \emph{magnitude} of forecast errors shrank after 1996.
 


%%%%%%%%%%%%%%%%%% Section 3: Forecast Accuracy %%%%%%%%%%%%%%%%%%
\section{Federal Reserve Staff's Forecast Accuracy}\label{ForecastAcc}

How accurate are Fed Staff forecasts? We focus on Greenbook forecasts of the GNP/GDP price index forecasts. We choose this indicator of Federal Reserve forecast accuracy because there is a strong assumption that central bankers are primarily concerned with inflation \citep[e.g.][]{Cukierman1992,Mukherjee2008,Tillmann2008}. It is also the dominant measure of forecast errors used in the economics literature \citep[c.f.][]{Romer2000}. 

We measure accuracy by calculating {\bf{forecast error}} $E$ as the difference between the Greenbook inflation forecast $F$ for a given quarter $q$ and actual inflation $I$ as a proportion of actual inflation:
%
\begin{equation}
    E_{q} = \frac{F_{q} - I_{q}}{I_{q}}.
\end{equation}

This is different from the accuracy measure \cite{Frendreis2000} used in their preliminary examination of forecast errors. They averaged the absolute value of yearly inflation forecast errors over a 19 year period\footnote{i.e. $\frac{|F_{y} - I_{y}|}{19}$} to examine Federal Reserve accuracy. Their measure has a number of drawbacks. First, it does not give us any indication of the direction of the forecast error, which is crucial for examining possible partisan biases. In their comparison of CBO and administrations' forecasts they did use a simple dichotomous directional indicator of accuracy in a given year (i.e. a forecast greater than or less than the actual level). This does not give us a sense of the relative size of the errors and could easily amplify trivial results. Almost any forecast will be above or below the actual inflation level in all but the unusual cases where the forecasts exactly equal the actual inflation level. 

Second, the average of the absolute errors values could be highly skewed by years of unusually large errors, which is more likely in years of higher inflation. This is not a trivial concern because the inflation level varies substantially overtime (see Figure \ref{absolute}).\footnote{\cite{Frendreis2000} also do not include any other indication of the errors' distribution.} So, we choose to focus on proportional rather than absolute errors by quarter to avoid focusing on a parameter that is highly vulnerable to absolute value outliers. Quarterly proportional errors are also more substantively meaningful for comparing errors across time periods.\footnote{Note that the direction and significance of our main findings do not change when we use absolute rather than proportional errors in our estimation models (discussed below). The magnitude does change, but this is to be expected because the range of the absolute inflation errors is much larger than proportional errors. These results are available from the authors upon request.} 

Third, using multi-year or even year-level indicators makes it difficult to examine biases in the run up to an election or any other process that may be observed through variations within years. Using quarterly data--the smallest level available--gives us a much more detailed view of any processes that might influence accuracy.

If the forecasts are unbiased the mean error of the forecasts--using either \cite{Frendreis2000} or our measure--would be indistinguishable from zero. While \cite{Frendreis2000} found that Fed errors were low relative to presidential administrations' on average over a 19 year period and \cite{Romer2000} found that the Fed's internal forecasts  meet this requirement over the full history of Greenbook forecasts, such an amalgamation disguises long periods of over- or under-predicting inflation, as noted in \cite{Capistran2006} and illustrated in Figure \ref{absolute}. Within economics the Fed's forecasts have been examined for evidence of rationality. These studies generally find that the Fed rationally incorporates information into its forecasts, outperforming private forecasts \cite[c.f.][]{Gamber2009}. These studies, however, have rarely incorporated Fed Staff member' political preferences, because Federal Reserve Staffers are assumed to be politically independent.

Our dataset has 169 forecast quarters,\footnote{This is the maximum number of observations. Longer forecasts result in fewer forecasted quarters. Likewise, some forecast lengths are unavailable for the full time period.} spanning the fourth quarter of 1965 through the end of 2007. Greenbook forecasts correspond to those provided for the FOMC meeting closest to the middle of the quarter. We found actual inflation corresponding to each of these quarters\footnote{Inflation was calculated by comparing quarters year-on-year. The exact inflation measure that the Federal Reserve was forecasting changed a number of times, so the measure of actual inflation used to created the forecast error variable changes accordingly. The GNP deflator indicator is used from the beginning of our sample through the end of 1991. From the first quarter of 1992 through the first quarter of 1996 actual inflation is measured with the GDP deflator. From the second quarter of 1996 we use the chain-weighted GDP price index. For more details on how the forecasted quantity changed see the Greenbook data description file available at: \url{http://www.phil.frb.org/research-and-data/real-time-center/greenbook-data/philadelphia-data-set.cfm}. The Greenbook inflation forecast variable we used is called ``PGDPdot''.} using data from the Federal Reserve's FRED website.\footnote{See \url{http://research.stlouisfed.org/fred2/}. Accessed December 2011.} We examine errors made by forecasters in the current quarter and all quarters up to five quarters before.\footnote{The Greenbook contains very incomplete data for forecasts made over longer time spans.} The results are generally the same regardless of the forecast's age, e.g. the results were similar for predictions made $q - 1$ quarters before the forecasted quarter $q$, $q - 2$ quarters before, and so on. In particular the presidential partisan findings are robust regardless of forecast age (see Figure \ref{ExpectValueParty}). For simplicity, the majority of results we show and discuss in detail are from models with forecasts made two quarters beforehand.\footnote{Using these two quarter forecasts restricts our observations because they are rarely available before the 1970s.} Figure \ref{absolute} compares absolute actual inflation for each quarter and inflation forecasts made two quarters before.

%%%%%%%%%%%%%%%%%%%%%%%%%%  Raw Greenbook estimate vs. actual graph
\begin{figure}[t]
    \caption{Greenbook Inflation Forecasts Made 2 Qtr. Beforehand and Actual Quarterly Inflation}
    \label{absolute}
    \begin{center}
    
\begin{knitrout}
\definecolor{shadecolor}{rgb}{0.969, 0.969, 0.969}\color{fgcolor}\begin{kframe}


{\ttfamily\noindent\bfseries\color{errorcolor}{\#\# Error: Could not resolve host: raw.github.com}}\end{kframe}
\end{knitrout}

    
    \end{center}
    \begin{singlespace}
        {\scriptsize{The vertical grey dotted line indicates when the Federal Reserve Board Global (FRB/Global) forecasting model was fully implemented.  
                      }}
    \end{singlespace}
\end{figure}


\subsection{Are There Partisan Forecast Errors?}

\begin{knitrout}
\definecolor{shadecolor}{rgb}{0.969, 0.969, 0.969}\color{fgcolor}\begin{kframe}


{\ttfamily\noindent\bfseries\color{errorcolor}{\#\# Error: Could not resolve host: raw.github.com}}\end{kframe}
\end{knitrout}


Unbiased forecasts have a mean error of zero \citep[5]{Bruck2006}. Using this criteria, forecast errors should be the same--ideally with a mean of 0--regardless of the incumbent president's party identification. This is not the case. From the second quarter of 1969\footnote{Data availability for two quarter forecasts before 1969 is lacking.} through 2007 the mean standardized forecast error was 

{\ttfamily\noindent\bfseries\color{errorcolor}{\\Error in mean(cpi.data\$error.prop.deflator.q2, na.rm = TRUE) : \\\ \ object 'cpi.data' not found}}, i.e. forecasters under-predicted inflation by about 

{\ttfamily\noindent\bfseries\color{errorcolor}{\\Error in mean(cpi.data\$error.prop.deflator.q2, na.rm = TRUE) : \\\ \ object 'cpi.data' not found}} percent. Our finding of relatively small average error over the entire 35 year period is in line with findings from previous studies. However, the mean errors are noticeably different across Republican compared to Democratic presidencies. Across Republican presidencies it was 

{\ttfamily\noindent\bfseries\color{errorcolor}{\\Error in eval(expr, envir, enclos) : object 'RepMean' not found}} percent and +

{\ttfamily\noindent\bfseries\color{errorcolor}{\\Error in eval(expr, envir, enclos) : object 'DemMean' not found}} percent across Democratic presidencies.\footnote{These means are from estimates made two quarters beforehand. Both means are statistically significantly different from 0, the full observation mean, and each other at the 99\% confidence level. For more details see \url{http://bit.ly/WHsRYh}.} On average, inflation was underestimated in Republican presidencies and overestimated in Democratic ones.

Figure \ref{errors_over_time} plots forecast errors across our sample separated by presidential term and party. The first thing to note is that inflation was rarely underestimated during the three Democratic presidential terms in our sample. The underestimates that were made were relatively small. The largest overestimates we see were made during Bill Clinton's (Democratic) presidency. All of the major inflation underestimates were made during Republican presidencies, particularly during Richard Nixon's, Gerald Ford's, and George W. Bush's presidencies. Inflation was often overestimated during the second part of Reagan's first term, his second term, and George H.W. Bush's term. Over this period--often referred to as the Volcker Revolution \citep{Bartels1985}--inflation was suddenly much lower than before (see Figure \ref{absolute}). It may have taken awhile for forecasters to adjust to this new lower level of inflation, particularly because the Fed's own models of the economy assumed that money had no real effects on the economy during this period, even while the FOMC was pursuing aggressive anti-inflation policies.

This summary examination of inflation forecast errors suggests that there may be a presidential partisan bias. Above we posited three different theories of how partisanship might affect inflation forecast errors. In the next section we describe how we go about testing these competing hypotheses.

%%%%%%%%%%%%%%%%%%%%%%%%%%   Greenbook Error Across Time
\begin{figure}[t]
    \caption{Errors in Inflation Forecasts Made 2 Qtr. Beforehand (1969 - 2007)}
    \label{errors_over_time}
    \begin{center}
    
\begin{knitrout}
\definecolor{shadecolor}{rgb}{0.969, 0.969, 0.969}\color{fgcolor}\begin{kframe}


{\ttfamily\noindent\bfseries\color{errorcolor}{\#\# Error: Could not resolve host: raw.github.com}}\end{kframe}
\end{knitrout}

    
    \end{center}
    \begin{singlespace}
        {\scriptsize{Note: An error of 0 indicates that inflation was perfectly predicted.}}
    \end{singlespace}
\end{figure}



%%%%%%%%%%%%%%%%%% Section 4: Research Methods %%%%%%%%%%%%%%%%%%

\section{Parametric Models \& Variables}

We used parametric models to examine the effects of presidential party ID and elections on the continuous inflation forecast error variable.\footnote{Parametric models are estimated using the \texttt{R} package \texttt{Zelig} \citep{Zelig2012}.} Our main model type was normal linear regression using maximum likelihood estimation of variance.\footnote{In {\tt{Zelig}} this is the {\tt{normal}} model.} To examine if our estimates were dependent on this model type we also ran our analyses ordinary least squares\footnote{In \texttt{Zelig}, this is the \texttt{ls} model). Because the results were virtually identical, we do not show them below. They are available upon request.} and Bayesian normal linear regression.\footnote{In {\tt{Zelig}} this is the {\tt{normal.bayes}} model.} Bayesian normal linear regression is particularly useful for our limited sample as it makes ``valid small sample inferences via draws from the exact posterior" \citep[][38]{Zelig2012}.\footnote{Please see \cite{Goodrich2007} for details about Bayesian normal linear regression.} 

As we show below the estimates from all three model types were very similar in direction, magnitude, and `statistical significance'. Estimates from each parametric model type were substantively identical.

\subsection{Variables}

In Section \ref{ForecastAcc} we discussed our dependent variable--inflation forecast errors. To examine possible partisan biases we are interested in whether US presidents' partisan identities and/or the existence of an upcoming presidential elections affect these errors. To do this we created {\bf{president party identification}} and {\bf{election period}} variables. The president party ID variable is 1 when the president is a Democrat and 0 when he is a Republican. Since forecast error data is released on a quarterly basis, we consider a president to be sitting from the first quarter after the election.\footnote{Elections are held almost at the midpoint--early November--of an election year's fourth quarter. Presidents are sworn into office near the beginning--20 January--of the following year's first quarter.} We consider quarters to be in the election period either if the presidential election is held in that quarter or in the previous three quarters.\footnote{If $q_{e}$ is a quarter with an election then we code quarters $q_{e}$, $q_{e-1}$, $q_{e-2}$, and $q_{e-3}$ election quarters.} The economic voting literature indicates that it is economic performance in the 6-12 months preceding an American presidential election that seem to matter most for the election's outcome \citep[c.f][]{Gelman1993}. 

To further examine whether or not Federal Reserve Staff were taking into consideration an electoral business cycle either due to a partisan preference or the nearness of an election, we include a variable of the {\bf{quarters until the presidential election}}. This simply counts down from the quarter after the previous election.\footnote{There are 15 quarters before a United States presidential election quarter.} The quarters that included presidential elections are coded as 0. This is used only in models testing for a partisan effect.

The partisan preference and monetary expectations theories both posit that president's party ID and elections have a non-linear interactive relationship with forecast errors. To examine this possibility we include an \textbf{interaction} between the president party ID variable and the square of the quarters until election variable in the analyses.\footnote{An interaction with the non-squared quarters until election variable is also included, following \citep{Brambor2006}.}

United States presidents do not set the level of government expenditure--a major non-monetary policy source of inflation--by themselves. Instead, presidents are constrained by the two houses of Congress. To examine whether or not Federal Reserve Staff are taking into consideration the partisan composition of Congress as well as presidents' party identifications, we include a variable measuring {\bf{Democratic legislators as a proportion of Republican legislators}} in the House of Representatives and a similar variable for the composition of the Senate.\footnote{Data on the number of legislators with Republican and Democratic party IDs was found at infoplease. See {\url{http://www.infoplease.com/ipa/A0774721.html}}. Accessed May 2012.} 

Because each chamber of Congress acts as a veto player on the main fiscal expenditures, any Congressional effect on errors likely works through an \textbf{interaction} between the partisan IDs of Congress and the presidency. There are two types of interaction effects that can be derived from the literature. The first interaction possibility is that Federal Reserve Staff, using simple rational partisan expectations, presume that a Democratic president would be able to get policies closer to their ideal point when there is a Congress with similar preferences. If a Democratic president faced chambers of Congress controlled by Democrats, presumably Federal Reserve Staff would expect even higher fiscal expenditures and therefore even higher inflation. Conversely, Republican presidents with a Republican-controlled Congress may be even better at cutting spending, leading to even lower inflation.\footnote{The inflationary effect of these policies may be mitigated if they were offset by higher or lower taxes respectively.}

The second possibility is based largely on Krause's \citeyearpar{Krause2000} work on the effect of partisan divisions on fiscal deficits in the United States. He finds partisan fragmentation can play a role in increasing federal deficits. Higher political conflict, he argues, ``results in equilibrium fiscal outcomes that favor greater spending and/or a willingness to lower taxes since politicians will exhibit a greater proclivity in providing voters with program benefits and to delay its payment" \citep[][542]{Krause2000}. Because of this Federal Reserve Staff may anticipate higher government borrowing when the presidency and houses of Congress are controlled by different parties. We are therefore agnostic about the theoretical direction of this interaction.

If prediction errors are largely the result of systematically biased economic forecasting models we would expect errors to change when the models changed. In particular, we would expect a decrease in the magnitude of errors around 1996 when the Federal Reserve Board's new US and Global Behavioral Equation Models were introduced. To examine this we include a {\bf{FRB/Global Model}} dummy variable. It equals one for all quarters from the first quarter of 1996 onward. It is zero otherwise.

Greenbook forecasts are based on the assumption that monetary policy will not change between when the prediction is made and the time period it is predicting.\footnote{While the Fed Staff also produce forecasts under alternative monetary policies in the so-called ``Bluebook," these data are not available in a readily usable format (i.e., not in a dataset but only in the original reports themselves) and thus are not used in the forecasting error literature.} However, since these forecasts are used in the setting of interest rates, this assumption often does not hold and forecast errors may occur if monetary policy changes in the interim. If this is the case monetary policy changes would have a negative relationship with forecast errors. When the FOMC raises interest rates inflation may decline, causing the original forecast to have been too high and vice versa. To control for monetary policy changes we include a variable of {\bf{standardized changes to the discount rate}} from the quarter the Greenbook prediction was made to the quarter it is predicting.\footnote{We averaged the discount rate over each quarter. Then we used the average discount rate $D$ in each quarter $q$ to create the variable $\Delta D_{q}$ using the simple formula: $\Delta D_{q} = \frac{D_{q} - D_{q-2}}{D_{q}}$. Note that the Federal Reserve changed how it used the discount rate and referred it at the beginning of 2003. To address this issue we primarily used data on the United States' discount rate recorded by the International Monetary Fund. Their data only goes back to the fourth quarter of 1982. So, before that we use the Federal Reserve's measure of the discount rate. Both of these variables are found in the FRED database at the St. Louis Federal Reserve (accessed July 2012). } The discount rate is one of the Federal Reserve's main tools for influencing the interest rate, especially the Fed Funds rate.\footnote{A similar {\bf{relative changes in the Fed Funds rate}} variable was included in some preliminary analyses. However, it did not change the results substantially and was estimated to have a similar effect on errors as the discount rate variable.}

We included a number of variables to examine if Federal Reserve inflation forecast errors are affected by incorrect assumptions about how levels of government expenditure impact inflation. These variables are the percentage of {\bf{current government expenditure to GDP}}, {\bf{government debt to GDP}},\footnote{Results for debt to GDP are not shown because it was never statistically significant in any of the models.} and \textbf{deficit to GDP}. Expenditure and debt are on a quarterly basis, while federal deficits are measured annually.\footnote{All three of these variables are from the FRED database, accessed October 2012 and January 2013.}

To examine how broader economic factors may be related to forecast errors we include variables of the {\bf{GDP output gap}}, {\bf{unemployment rate}} and {\bf{recession}}. The GDP output gap is the potential GDP as a percentage of real GDP. It is in nominal terms. The recession variable is a dummy for whether or not the United States was in a recession.\footnote{All three of these variables are from the FRED database, accessed June and October 2012.} 

Finally, we include a series of dummies for the sitting {\bf{Federal Reserve Board Chair}}.\footnote{Chairs for the years in our analysis are William McChesney Martin, Jr., Arthur Burns, G. William Miller, Paul Volcker, Alan Greenspan, and Ben Bernanke.}

Further variables used to examine omitted variable bias are discussed in the paper's Supplementary Materials.

\section{Results}

In this section we graphically present results from a number of parametric model specifications and discuss our findings. Full coefficient estimate tables can be found in tables \ref{OutputNL} and \ref{OutputNB}. There is little difference between the coefficients estimated using normal linear and Bayesian linear regression models (see Figure \ref{CoefComparePlots}).\footnote{In all of the Bayesian regressions we use the {\tt{Zelig}} default 1,000 MCMC burn-in iterations and 10,000 iterations after burn-in. We use the Heidelberger-Welch diagnostic to examine whether or not the Markov Chains converged to their stationary distributions.}

We remove quarters from the sample where forecasters would not have known who the president would be because the president had not yet been elected for that quarter. For models where the dependent variable is forecasts made two quarters beforehand this means removing the first two quarters of each presidential term.\footnote{In this case 19 quarters are removed.} Results from these restricted data sets are fairly similar to those from the full data set.





































